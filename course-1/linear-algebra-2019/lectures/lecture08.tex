\section{Лекция 2.11.2019}


\subsection{Следствия из критерия обратимости квадратной матрицы}

\begin{corollary}
    Если $AB = E$, то $BA = E$ (и тогда $A = B^{-1}$, $B = A^{-1}$).
\end{corollary}

\begin{proof}
    \label{proof:ab_inverse}
    \begin{equation*}
        AB = E \implies \det A \det B = 1 \implies \det A \neq 0 \implies \exists A^{-1}
    .\end{equation*}
    \begin{equation*}
        BA = EBA = (A^{-1}A)BA = A^{-1}(AB)A = A^{-1}A = E
    .\end{equation*}
\end{proof}

\begin{corollary}
    $A, B \in M_n \implies AB$ обратима $\iff$ обе $A$, $B$ обратимы. При этом $(AB)^{-1} = B^{-1} A^{-1}$.
\end{corollary}

\begin{proof}
    Эквивалентность ($\iff$) следует из условия $\det AB = \det A \det B$.

    \begin{equation*}
        (AB)(B^{-1}A^{-1}) = A(BB^{-1})A^{-1} = A A^{-1} = E
    .\qedhere\end{equation*}
\end{proof}

\subsection{Формулы Крамера}

Пусть есть СЛУ $Ax = b (\star)$, $A \in M_n$, $x = \begin{pmatrix} x_1 \\ \dots \\ x_n \end{pmatrix} \in \RR^n$, $b = \begin{pmatrix} b_1 \\ \dots \\ b_n \end{pmatrix} \in \RR^n$.

Также, $\forall i \in \{1, 2, \dots, n\}$, $A_i = (A^{(1)}, \dots, A^{(i - 1)}, b, A^{(i + 1)}, \dots, A^{(n)})$.

\begin{theorem}
    Если $\det A \neq 0$, то СЛУ ($\star$) имеет единственное решение и его можно найти по формулам:
    \begin{equation*}
        x_i = \frac{\det A_i}{\det A}
    .\end{equation*}
\end{theorem}

\begin{proof}
    $\det A \neq 0 \implies \exists A^{-1} \implies (\star) \iff x = A^{-1}b$ -- единственное решение.

    \begin{equation*}
        b = A \begin{pmatrix} x_1 \\ \vdots \\ x_n \end{pmatrix} = x_1 A^{(1)} + x_2 A^{(2)} + \dots + x_n A^{(n)}
    .\end{equation*}
    \begin{align*}
        \det A_i &= \det \left(A^{(1)}, \dots, A^{(i - 1)}, x_1 A^{(1)} + \dots + x_n A^{(n)}, A^{(i + 1)}, \dots, A^{(n)}\right) \\
                 &= x_1 \det \left(A^{(1)}, \dots, A^{(i - 1)}, A^{(1)}, A^{(i + 1)}, \dots A^{(n)}\right) \\
                 & \quad + x_2 \det \left(A^{(1)}, \dots, A^{(i - 1)}, A^{(2)}, A^{(i + 1)}, \dots, A^{(n)}\right) \\
                 & \quad + \dots + \\
                 & \quad+ x_n \det \left(A^{(1)}, \dots, A^{(i - 1)}, A^{(n)}, A^{(i + 1)}, \dots, A^{(n)}\right) \\
                 &= x_i \det A \quad \text{ /\!/ Все слагаемые кроме i-го равны 0}
    .\qedhere\end{align*}
\end{proof}


\subsection{Понятие поля.}

\begin{definition}
    \textit{Полем} называется множество $F$, на котором заданы две операции ``сложение'' ($(a, b) \to a + b$) и ``умножение'' ($(a, b) \to a \cdot b$), причем $\forall a, b, c \in F$ выполнены следующие условия:

    \begin{enumerate}[nosep]
        \item $a + b = b + a$ (коммутативность сложения)
        \item $(a + b) + c = a + (b + c)$ (ассоциативность сложения)
        \item $\exists 0 \in F : 0 + a = a + 0 = a$ (нулевой элемент)
        \item $\exists (-a) \in F: a+(-a)=(-a)+a=0$ (противоположный элемент)

            $\uparrow$ абелева группа $\uparrow$
        \item $a(b+c) = ab + ac$ (дистрибутивность)
        \item $ab=ba$ (коммутативность умножения)
        \item $(ab)c=a(bc)$ (ассоциативность умножения)
        \item $\exists 1 \in F \setminus \{0\} : 1 a = a 1 = a$ (единица)
        \item Если $a \neq 0$, $\exists a^{-1} \in F : a a^{-1} = a^{-1} a = 1$ (обратный элемент)
    \end{enumerate}
\end{definition}


\subsection{Простейшие примеры.}

$\QQ$ -- Рациональные числа.

$\RR$ -- Действительные числа.

$F_2 = \{0, 1\}$, сложение и умножение по модулю 2.


\subsection{Построение поля комплексных чисел.}

Ближайшая цель --- построить поле $\CC$ комплексных чисел.

Неформально, $\CC$ -- это наименьшее поле со следующими свойставми:
\begin{enumerate}
\item $\CC \supset \RR$.
\item Многочлен $x^2 + 1$ имеет корень, то есть $\exists i : i^2 = -1$.
\end{enumerate}


\subsubsection{Формальная конструкция поля $\CC$}

\begin{equation*}
    \CC = \RR^2 = \{(a, b) \mid a, b \in \RR\}
.\end{equation*}

\begin{itemize}
\item $(a_1, b_1) + (a_2, b_2) = (a_1 + a_2, b_1 + b_2)$
\item $(a_1, b_1) (a_2, b_2) = (a_1 a_2 - b_1 b_2, a_1 b_2 + a_2 b_1)$
\end{itemize}

Неформально, каждой такой паре $(a, b)$ соответствует комплексное число $a + bi$:
\begin{itemize}
\item $(a, b) \iff a + bi$
\item $(a_1 + b_1 i) + (a_2 + b_2 i) = (a_1 + a_2) + (b_1 + b_2)i$
\item $(a_1 + b_1 i) (a_2 + b_2 i) = a_1 a_2 + a_1 b_2 i + a_2 b_1 i + b_1 b_2 \underbrace{i^2}_{= -1} = (a_1 a_2 - b_1 b_2) + (a_1 b_2 + a_2 b_1) i$
\end{itemize}

\subsubsection{Проверка аксиом}

\begin{enumerate}
\item[1, 2.] Очевидны.
\setcounter{enumi}{2}
\item $0 = (0, 0)$.
\item $-(a, b) = (-a, -b)$.
\item Дистрибутивность
    \begin{align*}
        (a_1 + b_1 i) ((a_2 + b_2 i) + (a_3 + b_3 i))
        &= (a_1 + b_1 i) ((a_2 + a_3) + (b_2 + b_3) i) \\
        &= (a_1 (a_2 + a_3) - b_1 (b_2 + b_3)) + (a_1 (b_2 + b_3) + b_1 (a_2 + a_3)) i \\
        &= a_1 a_2 + a_1 a_3 - b_1 b_2 - b_1 b_3 + (a_1 b_2 + a_1 b_3 + b_1 a_2 + b_1 a_3) i \\
        &= ((a_1 a_2 - b_1 b_2) + (a_1 b_2 + b_1 a_2)i) + ((a_1 a_3 + b_1 b_3) + (b_1 a_3 + a_1 b_3) i) \\
        &= (a_1 + b_1 i)(a_2 + b_2 i) + (a_1 + b_1 i)(a_3 + b_3 i)
    \end{align*}
\item Коммутативность умножения -- из явного вида формулы.
    \begin{equation*}
        (a_1 + b_1 i) (a_2 + b_2 i) = (a_1 a_2 - b_1 b_2) + (a_1 b_2 + a_2 b_1) i
    \end{equation*}

\item Ассоциативность умножения
    \begin{align*}
        (a_1, b_1)(a_2, b_2)(a_3, b_3)
        &= (a_1 a_2 - b_1 b_2, a_1 b_2 + a_2 b_1) (a_3, b_3) \\
        &= (a_1 a_2 a_3 - b_1 b_2 a_3 - a_1 b_2 b_3 - b_1 a_2 b_3, a_1 a_2 b_3 - b_1 b_2 b_3 + a_1 b_2 a_3 + b_1 a_2 a_3) \\
        &= (a_1, b_1)(a_2 a_3 - b_2 b_3, a_2 b_3 + b_2 a_3) \\
        &= (a_1, b_1)(a_2, b_2)(a_3, b_3)
    .\end{align*}

\item $1 = (1, 0)$.

\item $(a, b) \neq 0 \implies a^2 + b^2 \neq 0$. Тогда, $(a, b)^{-1} = \left(\frac{a}{a^2 + b^2}, -\frac{b}{a^2 + b^2}\right)$.

    $(a, b) \left(\frac{a}{a^2 + b^2}, \frac{-b}{a^2 + b^2}\right) = \left(\frac{a^2}{a^2 + b^2} + \frac{b^2}{a^2 + b^2}, \frac{-ab}{a^2 + b^2} + \frac{ba}{a^2 + b^2}\right) = (1, 0)$.
\end{enumerate}

Итак, $\CC$ -- поле.

\paragraph{Проверка свойств}
\begin{enumerate}
\item $a \in \RR \leftrightarrow (a, 0) \in \CC$.

    $a + b \leftrightarrow (a, 0) + (b, 0) = (a + b, 0)$.

    $ab \leftrightarrow (a, 0)(b, 0) = (ab, 0)$

    Значит, $\RR$ отождествляется в $\CC$.

\item
    $i = (0, 1) \implies i^2 = (0, 1)(0, 1) = (-1, 0) = -1$.
\end{enumerate}


\subsection{Алгебраическая форма комплексного числа, его действительная и мнимая части.}

\begin{definition}
    Представление числа $z \in \CC$ в виде $a + bi$, где $a, b \in \RR$ называется его \textit{алгебраической формой}.
    Число $i$ называется \textit{мнимой единицей}.

    $a =: Re(z)$ -- \textit{действительная} часть числа $z$.

    $b =: Im(z)$ -- \textit{мнимая} часть числа $z$.
\end{definition}

Числа вида $bi$, где $b \in \RR \setminus \{0\}$, называются \textit{чисто мнимыми}.

\subsection{Комплексное сопряжение.}

\begin{definition}
    Число $\overline{z} := a - bi$ называется \textit{комплексно сопряженным} к числу $z = a + bi$.

    Операция $z \to \overline{z}$ называется \textit{комплексным сопряжением}.
\end{definition}

\subsubsection{Свойства комплексного сопряжения}

\begin{itemize}[nosep]
\item $\overline{\overline{z}} = z$.
\item $\overline{z + w} = \overline{z} + \overline{w}$.
\item $\overline{zw} = \overline{z} \cdot \overline{w}$.
% TODO: Добавить про сопряжение деления? (Было доп вопросом на коллке).
\end{itemize}

\begin{proof}~
    \begin{itemize}
    \item $\overline{\overline{z}} = \overline{\overline{a + bi}} = \overline{a - bi} = a + bi = z$.
    \item $\overline{z + w} = \overline{(a_1 + b_1 i) + (a_2 + b_2 i)} = \overline{(a_1 + a_2) + (b_1 + b_2) i} = (a_1 + a_2) - (b_1 + b_2)i = (a_1 - b_1 i) + (a_2 - b_2 i) = \overline{z} + \overline{w}$.
    \item $\overline{z} \cdot \overline{w} = (a_1 - b_1 i) (a_2 - b_2 i) = (a_1 a_2 - b_1 b_2) - (a_1 b_2 + a_2 b_1) i = \overline{zw}$. \qedhere
    \end{itemize}
\end{proof}

\subsection{Геометрическая модель комплексных чисел, интерпретация сложения и сопряжения в этой модели.}

Числу $z = a + bi$ соответствует точка (или вектор) на плоскости $\RR^2$ с координатами $(a, b)$.
Сумме $z + w$ соответствует сумма соответствующих векторов.
Сопряжение $z \to \overline{z}$ -- это отражение $z$ относительно действительной оси.
