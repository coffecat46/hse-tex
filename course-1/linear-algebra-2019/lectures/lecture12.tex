\section{Лекция 28.11.2019}

Пусть $V$ -- векторное пространство над полем $F$.

Обозначение $\dim V < \infty$ -- $V$ конечномерно.

\subsection{Характеризация базисов в терминах единственности линейного выражения векторов}

\begin{proposition}
    Пусть $\dim V < \infty$, $e_1, \dots, e_n \in \langle V \rangle$.

    $e_1, \dots, e_n$ --- базис $V$ тогда и только тогда, когда, $\forall v \in V$ единственным образом представим в виде
    \begin{equation*}
        v = x_1 e_1 + \dots + x_n e_n \quad x_i \in F
    .\end{equation*}
\end{proposition}

\begin{proof}~
    \begin{description}
    \item[$\implies$]
        Пусть есть два представления $v = x_1 e_1 + \dots x_n e_n = x'_1 e_1 + \dots + x'_n e_n$.

        Тогда, $(x_1 - x'_1)e_1 + \dots + (x_n - x'_n)e_n = \overrightarrow{0}$.

        Так как $e_1, \dots, e_n$ линейно независимы, то $(x_1 - x'_1) = \dots = (x_n - x'_n) = 0$.

        Значит, $x_i = x'_i \quad \forall i$.

    \item[$\impliedby$]
        $\forall v \in V$ имеем $v \in \langle e_1, \dots, e_n \rangle$.

        Значит, $\langle e_1, \dots, e_n \rangle = V$.

        Для $v = \overrightarrow{0}$ существует единственное представление $\overrightarrow{0} = \lambda_1 e_1 + \dots + \lambda_n e_n$.

        Но мы знаем, что $\overrightarrow{0} = 0e_1 + \dots + 0e_n$.

        Следовательно $\alpha_1 = \dots \alpha_n = 0$, то есть $e_1, \dots, e_n$ линейно независимо.
    \end{description}

    Итог: $e_1, \dots, e_n$ -- базис $V$.
\end{proof}


\subsection{Фундаментальная система решений однородной системы линейных уравнений}

\begin{equation*}
    \tag{$\star$}
    \label{lec12:oslu}
    Ax = 0 \text{ -- ОСЛУ}
.\end{equation*}

$A \in \text{Mat}_{m \times n}(F), x = \begin{pmatrix} x_1 \\ \dots \\ x_n \end{pmatrix} \in F^n$.

$S \subseteq F^n$ -- множество решений.

Знаем, что $S$ -- подпространство в $F^n$.

\begin{definition}
    \textit{Фундаментальной системой решений} (ФСР) для ОСЛУ \eqref{lec12:oslu} называется всякий базис пространства её решений.
\end{definition}

\begin{comment}
    У одной ОСЛУ может быть много разных ФСР.
\end{comment}


\subsection{Метод построения фундаментальной системы решений}

Приведем матрицу к улучшенному ступенчатому виду элементарными преобразовиями строк.
\begin{equation*}
    (A | \overrightarrow{0}) \rightsquigarrow (B | \overrightarrow{0}) \quad \leftarrow \text{ улучшенный ступенчатый вид}
.\end{equation*}

Пусть $r$ -- число ненулевых строк в $B$.

Тогда будет $r$ главных неизвестных и $n - r$ свободных.

Выполнив перенумерацию будем считать что,
\begin{align*}
    x_1, \dots, x_r &\text{ -- главные неизвестные,} \\
    x_{r + 1}, \dots, x_n &\text{ -- свободные}
.\end{align*}

Тогда, общее решение для $\eqref{lec12:oslu}$ имеет вид
\begin{align*}
    &x_1 = c_{11} x_{r + 1} + c_{12}x_{r + 2} + \dots + c_{1, n - r} x_n \\
    &x_2 = c_{21} x_{r + 1} + c_{22}x_{r + 2} + \dots + c_{2, n - r} x_n \\
    &\dots \\
    &x_r = c_{r1} x_{r + 1} + c_{r2}x_{r + 2} + \dots + c_{r, n - r} x_n
.\end{align*}

Предъявим некоторую систему решений
\begin{equation*}
    u_1 = \begin{pmatrix} 
        c_{11} \\
        c_{21} \\
        \dots \\
        c_{r1} \\
        \underline{1} \\
        0 \\
        \dots \\
        0
    \end{pmatrix}, \
    u_2 = \begin{pmatrix} 
        c_{12} \\
        c_{22} \\
        \dots \\
        c_{r2} \\
        0 \\
        \underline{1} \\
        0 \\
        \dots \\
        0
    \end{pmatrix}, \ \dots, \
    u_{n - r} = \begin{pmatrix} 
        c_{1, n - r} \\
        c_{2, n - r} \\
        \dots \\
        c_{r, n - r} \\
        0 \\
        \dots \\
        0 \\
        \underline{1}
    \end{pmatrix}
.\end{equation*}

$u_1, \dots, u_{n - r} \in S$

\begin{proposal}
    $u_1, \dots, u_{n - r}$ -- это ФСР для ОСЛУ \eqref{lec12:oslu}.
\end{proposal}

\begin{proof}~
    \begin{enumerate}
    \item Линейная независимость.

        Пусть $\alpha_1 u_1 + \dots + \alpha_{n - r} u_{n - r} = \overrightarrow{0}$.

        При любом $k \in \{1, \dots, n - r\}$, ($r + k$)-я координата левой части равна $\alpha_k$, значит $\alpha_k = 0$.
        
        Следовательно $\alpha_1 = \dots = \alpha_{n - r} = 0$.

    \item $\langle u_1, \dots, u_{n - r} \rangle = S$.

        \begin{description}
        \item[``$\subseteq$'']
            Верно, так как $u_1, \dots, u_{n - r} \in S$.

        \item[``$\supseteq$'']
            Пусть $u \in S$, тогда
            \begin{equation*}
                u = \begin{pmatrix} * \\ \dots \\ * \\ \alpha_1 \\ \alpha_2 \\ \dots \\ \alpha_{n - r} \end{pmatrix} \text{ для некоторых } \alpha_1, \dots, \alpha_{n - r} \in F
            .\end{equation*}

            Положим $v := u - \alpha_1 u_1 - \dots - \alpha_{n - r} u_{n - r}$.

            Тогда, $v \in S$, но
            \begin{equation*}
                v = \begin{pmatrix} * \\ \dots \\ * \\ 0 \\ \dots \\ 0 \end{pmatrix}
            .\end{equation*}

            Тогда формулы для общего решения дают $v = \overrightarrow{0}$.

            Поэтому $u = \alpha_i u_1 + \dots + \alpha_{n - r} u_{n - r}$.
        \end{description}
        Значит $\langle u_1, \dots, u_{n - r} \rangle = S$.
        \qedhere
    \end{enumerate}
\end{proof}

\begin{example}
    \begin{equation*}
        A = \begin{pmatrix} 
            1 & -3 & 0 & 1 \\
            0 & 0 & 1 & -2
        \end{pmatrix}
    .\end{equation*}

    Общее решение:
    \begin{equation*}
        \begin{cases}
            x_1 = 3x_2 - x_4 \\
            x_3 = 2x_4
        \end{cases}
    \end{equation*}

    Тогда ФСР:
    \begin{equation*}
        u_1 = \begin{pmatrix} 3 \\ 1 \\ 0 \\ 0 \end{pmatrix}, \
        u_2 = \begin{pmatrix} -1 \\ 0 \\ 2 \\ 1 \end{pmatrix}
    .\end{equation*}
\end{example}


\subsection{Утверждение о возможности выбора из конечной системы векторов базиса её линейной оболочки}

Пусть $V$ -- векторное пространство над F.

Наблюдение: если $v, v_1, \dots, v_m \in V$ и $v \in \langle v_1, \dots, v_m \rangle$, тогда $\langle v, v_1, \dots, v_m \rangle = \langle v_1, \dots, v_m \rangle$

\begin{proposal}
    Из всякой конечной системы векторов $S \subseteq V$ можно выбрать подсистему, которая является базисом в линейной оболочке $\langle S \rangle$.
\end{proposal}

\begin{proof}
    Пусть $S = \{v_1, \dots, v_m\}$.

    Индукция по $m$.

    \begin{description}
    \item[База]
        $m = 1$: $S = \{v_1\}$.

        Если $v_1 = \overrightarrow{0}$, то $\langle S \rangle = \{\overrightarrow{0}\}$, значит в качестве базиса берем $\varnothing$.

        Если $v_1 \neq 0$, то $S$ линейно независимо.

        Cледовательно $S$ -- базис в $\langle S \rangle$.

    \item[Шаг]
        Пусть доказано для $< m$, докажем для $m$.

        Если $v_1, \dots, v_m$ линейно независимо, то $v_1, \dots, v_m$ -- это уже базис в $\langle S \rangle$.

        Иначе, $\exists i : v_i \in \langle S \setminus \{v_i\} \rangle$.

        Положим $S' := S \setminus \{v_i\}$.

        Тогда, $\langle S' \rangle = \langle S \rangle$.

        Так как $|S'| = m - 1 < m$, то по предположению индукции в $S'$ можно выбрать базис для $\langle S' \rangle = \langle S \rangle$.
        \qedhere
    \end{description}
\end{proof}

\subsection{Дополнение конечной линейно независимой системы векторов до базиса конечномерного векторного пространства}

\begin{proposal}
    Пусть $\dim V < \infty$, тогда всякую линейно независимую систему векторов в $V$ можно дополнить до базиса всего пространства $V$.
\end{proposal}

\begin{proof}
    Пусть $v_1, \dots, v_m$ -- данная линейно независимая система.

    Так как $\dim V < \infty$, в $V$ есть конечный базис $e_1, \dots, e_n$.

    Рассмотрим систему векторов $v_1, \dots, v_m, e_1, \dots, e_n$.

    Пройдемся по этим векторам слева направо и выбросим те, которые линейно выражаются через предыдущие (не выброшенные).

    При этом:
    \begin{enumerate}[nosep, label=\arabic*)]
    \item линейная оболочка системы сохраняется и равна $\langle v_1, \dots, v_m, e_1, \dots, e_n \rangle = V$;
    \item $v_1, \dots, v_m$ останутся в системе, так как они линейно независимы;
    \item в новой системе никакой вектор линейно не выражается через предыдущие.
    \end{enumerate}

    Пусть новая система - это $S' = \{v_1, \dots, v_m, e_{i_1}, \dots, e_{i_t}\}$.

    Докажем, что $S'$ -- базис в $V$.

    По свойству 1) имеем, что $\langle S' \rangle = V$.

    Осталось доказать, что $S'$ линейно независимо.

    Пусть $\alpha_1 v_1 + \dots \alpha_n v_n + \beta_1 e_{i_1} + \dots + \beta_t e_{i_t} = \overrightarrow{0}$.

    Предположим, что эта линейная комбинация нетривиальна.

    Так как $v_1, \dots, v_m$ линейно независимы, то $\exists k : \beta_{i_k} \neq 0$.

    Выберем $k$ максимальным с этим свойством.

    Тогда, $e_{i_k}$ линейно выражается через предыдущие --- противоречие.
\end{proof}

\begin{corollary}
    Если $\dim V = n$ и $v_1, \dots, v_n$ -- линейно независимая система, тогда $v_1, \dots, v_n$ -- базис $V$.
\end{corollary}

\subsection{Лемма о добавлении вектора к конечной линейной независимой системе}

\begin{lemma}
    \label{lec12:lemma}
    Пусть $v, v_1, \dots, v_m \in V$ и $v_1, \dots, v_m$ линейно независимы, тогда либо $v, v_1, \dots, v_m$ линейно независимы, либо $v \in \langle v_1, \dots , v_m \rangle$.
\end{lemma}

\begin{proof}
    Пусть $v, v_1, \dots, v_m$ линейно зависимы, тогда $\exists (\alpha, \alpha_1, \dots, \alpha_m) \neq (0, \dots, 0)$, такой что
    \begin{equation*}
        \alpha v + \alpha_1 v_1 + \dots + \alpha_m v_m = \overrightarrow{0}
    .\end{equation*}
    Но, так как $v_1, \dots, v_m$ линейно независимы, то $\alpha \neq 0$. Значит, $v \in \langle v_1, \dots, v_m \rangle$ по \hyperref[lec11:proposal_a_i]{предложению}.
\end{proof}
