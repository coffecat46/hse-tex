\section{Лекция 19.03.2020} 

\href{https://www.dropbox.com/s/y3r4x7mjt7iv71a/%D0%9B%D0%90%D0%B8%D0%93_19-20_%D0%9B%D0%B5%D0%BA%D1%86%D0%B8%D1%8F_25.pdf?dl=0}{Записки с лекции}


\begin{lemma}
    Пусть $v_1, v_2 \in \EE$. Тогда, $(v_1, x) = (v_2, x) \ \forall x \in \EE \implies v_1 = v_2$.
\end{lemma}

\begin{proof}
    Имеем $(v_1 - v_2, x) = 0 \ \forall x \in \EE$.
    Тогда, $v_1 - v_2 \in \EE^{\perp} = \{0\} \implies v_1 - v_2 = 0 \implies v_1 = v_2$.
\end{proof}


\subsection{Трёхмерное евклидово пространство}

\begin{theorem}
    \label{lec25:t}
    Пусть $a, b \in \RR^3$. Тогда
    \begin{enumerate}
    \item $\exists! v \in \EE$, такой что $(v, x) = \Vol(a, b, x) \quad \forall x \in \RR^3$.
    \item Если $\E = (e_1, e_2, e_3)$ --- положительно ориентированный ортонормированный базис и
        \begin{math}
            \ \begin{aligned}[t]
                a &= a_1 e_1 + a_2 e_2 + a_3 e_3 \\
                b &= b_1 e_1 + b_2 e_2 + b_3 e_3
            \end{aligned},
        \end{math}
        то 
        \begin{equation}
            \tag{$\star$}
            \label{lec25:v}
            v = \begin{vmatrix}
                e_1 & e_2 & e_3 \\
                a_1 & a_2 & a_3 \\
                b_1 & b_2 & b_3
            \end{vmatrix}
            := \begin{vmatrix} 
                a_2 & a_3 \\
                b_2 & b_3
            \end{vmatrix} e_1 - \begin{vmatrix} 
                a_1 & a_3 \\
                b_1 & b_3
            \end{vmatrix} e_2 + \begin{vmatrix} 
                a_1 & a_2 \\
                b_1 & b_2
            \end{vmatrix} e_3
        .\end{equation}
    \end{enumerate}
\end{theorem}

\begin{proof}~
    \begin{description}
    \item[Единственность] если $v'$ --- другой такой вектор, то $(v, x) = (v', x) \ \forall x \in \RR^3$, а значит $v' = v$ по лемме.
    \item[Существование] Покажем, что $v$, заданный формулой \eqref{lec25:v} подойдёт.
        \begin{align*}
            x = x_1 e_1 + x_2 e_2 + x_3 e_3 \implies (v, x) &= \begin{vmatrix} 
                a_2 & a_3 \\
                b_2 & b_3
            \end{vmatrix} x_1 - \begin{vmatrix} 
                a_1 & a_3 \\
                b_1 & b_3
            \end{vmatrix} x_2 + \begin{vmatrix} 
                a_1 & a_2 \\
                b_1 & b_2
                \end{vmatrix} x_3 \\ &= \begin{vmatrix} 
                x_1 & x_2 & x_3 \\
                a_1 & a_2 & a_3 \\
                b_1 & b_2 & b_3
            \end{vmatrix} = \begin{vmatrix} 
                a_1 & a_2 & a_3 \\
                b_1 & b_2 & b_3 \\
                x_1 & x_2 & x_3
            \end{vmatrix} = \Vol(a, b, x)
        .\end{align*}
    \end{description}
\end{proof}


\subsection{Векторное произведение, его выражение в координатах}

\begin{definition}
    Вектор $v$ из теоремы выше называется \textit{векторным произведением} векторов $a$ и $b$.

    Обозначение: $[a, b]$ или $a \times b$.
\end{definition}


\subsection{Смешанное произведение трёх векторов, его свойства}

\begin{definition}
    $\forall a, b, c \in \EE$ число $(a, b, c) := ([a, b], c)$ называется \textit{смешанным произведением} векторов $a, b, c$.
\end{definition}

\begin{comment}
    Из \hyperref[lec25:t]{теоремы} видно, что $(a, b, c) = \Vol(a, b, c)$.
\end{comment}

\begin{proof}[Свойства смешанного произведения]~
    \begin{enumerate}[nosep]
    \item 
        $(a, b, c) > 0 \iff a, b, c$ --- положительно ориентированный базис,
        
        $(a, b, c) < 0 \iff a, b, c$ --- отрицательно ориентированный базис.

        \medskip
        Критерий компланарности ($= $ линейной зависимости)
        \begin{equation*}
            a, b, c\text{ компланарны} \iff (a, b, c) = 0
        .\end{equation*}

    \item Линейность по каждому аргументу.

    \item Кососимметричность (меняет знак при перестановке любых двух векторов).

    \item Если $e_1, e_2, e_3$ --- положительно ориентированный ортонормированный базис, то
        \begin{equation*}
            \left.\begin{aligned}
                a &= a_1 e_1 + a_2 e_2 + a_3 e_3 \\
                b &= b_1 e_1 + b_2 e_2 + b_3 e_3 \\
                c &= c_1 e_1 + c_2 e_2 + c_3 e_3
            \end{aligned} \right| \implies (a, b, c) = \begin{vmatrix} 
                a_1 & a_2 & a_3 \\
                b_1 & b_2 & b_3 \\
                c_1 & c_2 & c_3
            \end{vmatrix}
        \end{equation*}
    \end{enumerate}
\end{proof}


\subsection{Критерий коллинеарности двух векторов в терминах векторного произведения}

\begin{proposal}
    $a, b \in \EE$ коллинеарны $\iff [a, b] = 0$.
\end{proposal}

\begin{proof}~
    \begin{description}
        \item[$\implies$] 
            \begin{equation*}
                (a, b, x) = 0 \ \forall x \implies ([a, b], x) = 0 \ \forall x \implies [a, b] = 0
            .\end{equation*}

        \item[$\impliedby$]
            \begin{equation*}
                [a, b] = 0 \implies ([a, b], x) = 0 \ \forall x \implies (a, b, x) = 0 \ \forall x \in \RR^3
            .\end{equation*}

            Если $a$, $b$ линейно независимы, то можно взять $x$, который дополняет их до базиса в $\RR^3$.

            Тогда, $(a, b, x) \neq 0$ --- противоречие. Значит $a$, $b$ линейно зависимы $\implies$ коллинеарны.
            \qedhere
    \end{description}
\end{proof}


\subsection{Геометрические свойства векторного произведения}

\begin{proposal}~
    \begin{enumerate}[nosep]
    \item $[a, b] \perp \left< a, b \right>$.
    \item $\left|[a, b]\right| = \vol P(a, b)$.
    \item $\Vol(a, b, [a, b]) \geq 0$.
    \end{enumerate}
\end{proposal}

\begin{proof}~
    \begin{enumerate}
    \item $([a, b], a) = (a, b, a) = 0 = (a, b, b) = ([a, b], b)$.
    \item Если $a$, $b$ коллинеарны, то обе части равны 0.

        Пусть $[a, b] \neq 0$.
        \begin{equation*}
            \left|[a, b]\right|^2 = ([a, b], [a, b]) = (a, b, [a, b]) = (\#) > 0
        .\end{equation*}
        \begin{equation*}
            [a, b] \perp \left< a, b \right> \implies (\#) = \vol P(a, b, [a, b]) = \Vol(a, b, [a, b]) = \vol P(a, b,) \cdot \left|[a, b]\right|
        .\end{equation*}

        Сокращая на $|[a, b]| \neq 0$, получаем требуемое.

    \item
        $\Vol (a, b, [a, b]) = ([a, b], [a, b]) \geq 0$.
        \qedhere
    \end{enumerate}
\end{proof}

\begin{exercise}
    $[a, b]$ однозначно определяется свойствами $1)$ --- $3)$.
\end{exercise}

\subsection{Антикоммутативность и билинейность векторного произведения}

\begin{example}
    Пусть $e_1, e_2, e_3$ --- положительно ориентированный ортонормированный базис в $\RR^3$.
    \begin{table}[h]
        \centering
        \begin{tabular}{c|c|c|c|}
            $[e_i, e_j]$ & $e_1$ & $e_2$ & $e_3$ \\ \hline
            $e_1$ & $0$ & $e_3$ & $-e_2$ \\ \hline
            $e_2$ & $-e_3$ & $0$ & $e_1$ \\ \hline
            $e_3$ & $e_2$ & $-e_1$ & $0$ \\ \hline
        \end{tabular}
    \end{table}
\end{example}

\begin{proposal}~
    \begin{enumerate}
    \item $[a, b] = -[b, a] \quad \forall a, b$ (антикоммутативность).
    \item $[\bigcdot, \bigcdot]$ билинейно (то есть линейно по каждому аргументу).
    \end{enumerate}
\end{proposal}

\begin{proof}~
    \begin{enumerate}
    \item 
        $([a, b], x) = (a, b, x) = -(b, a, x) = -([b, a], x) = (-[b, a], x) \quad \forall x \in \RR^3 \implies [a, b] = -[b, a]$

    \item
        Пусть $u = [\lambda_1 a_1 + \lambda_2 a_2, b]$, $v = \lambda_1 [a_1, b] + \lambda_2 [a_2, b]$.
        Тогда $\forall x \in \RR^3$:
        \begin{align*}
            (u, x) &= (\lambda_1 a_1 + \lambda_2 a_2, b, x) \\
                   &= \lambda_1 (a_1, b, x) + \lambda_2 (a_2, b, x) \\
                   &= \lambda_1 ([a_1, b], x) + \lambda_2([a_2, b], x) \\
                   &= (\lambda_1[a_1, b] + \lambda_2[a_2, b], x) = (v, x)
        .\end{align*}

        Значит $u = v$. Аналогично линейность по второму аргументу.
        \qedhere
    \end{enumerate}
\end{proof}


\subsection{Линейные многообразия в $\RR^n$}

\begin{definition}
    \textit{Линейное многообразие} в $\RR^n$ --- это множество решений некоторой совместной СЛУ.
\end{definition}


\subsection{Характеризация линейных многообразий как сдвигов подпространств}

Пусть $Ax = b$ --- СЛУ, $\varnothing \neq L \subseteq \RR^n$ --- множество решений, $x_z \in L$ --- частное решение.

Было: Лемма: $L = x_z + S$, где $S$ --- множество решений ОСЛУ $Ax = 0$.

\begin{proposal}
    Множество $L \subseteq \RR^n$ является линейным многообразием $\iff L = v_0 + S$ для некоторых $v_0 \in \RR^n$ и подпространства $S \subseteq \RR^n$. 
\end{proposal}

\begin{proof}~
    \begin{description}
    \item[$\implies$] Из леммы.
    \item[$\impliedby$] $L = v_0 + S$. Значит существует ОСЛУ $Ax = 0$, для которой $S$ является множеством решений. Тогда, $L$ --- множество решений СЛУ $Ax = Av_0$ (по лемме).
        \qedhere
    \end{description}
\end{proof}


\subsection{Критерий равенства двух линейных многообразий}

\begin{proposal}
    Пусть $L_1 = v_1 + S_1$ и $L_2 = v_2 + S_2$ --- два линейных многообразия в $\RR^n$. Тогда,
    \begin{equation*}
        L_1 = L_2 \iff \begin{cases}
            S_1 = S_2 \ (= S) \\
            v_1 - v_2 \in S
        \end{cases}
    .\end{equation*}
\end{proposal}

\begin{proof}~
    \begin{description}
    \item[$\impliedby$] 
        $L_1 = v_1 + S_1 = v_1 + S_2 = v_2 + (v_1 - v_2) + S = v_2 + S = L_2$.
    \item[$\implies$]
        $v_1 = v_1 + 0 \in L_1 = L_2 = v_2 + S_2 \implies v_1 - v_2 \in S_2$,

        $v \in S_1 \implies v + v_1 \in L_1 = L_2 = v_2 + S_2 \implies v \in (v_2 - v_1) + S_2 = S_2 \implies S_1 \subseteq S_2$.

        Аналогично, $v_1 - v_2 \in S_1$ и $S_2 \subseteq S_1$.
        \qedhere
    \end{description}
\end{proof}


\subsection{Направляющее подпространство и размерность линейного многообразия}
 
Если $L$ --- линейное многообразие, то $L = v_0 + S$, где $S$ определено однозначно.

\begin{definition}
    $S$ называется \textit{направляющим подпространством} линейного многообразия $L$.
\end{definition}

\begin{definition}
    \textit{Размерностью} линейного многообразия называется размерность его направляющего подпространства.
\end{definition}

