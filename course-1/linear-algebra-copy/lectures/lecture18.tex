\section{Лекция 25.01.2020}

\subsection{Критерий инъективности линейного отображения в терминах его ядра}

Пусть $V, W$ --- векторные пространства над $F$, 

$\phi \colon V \to W$ --- линейное отображение.

Ядро: $\ker \phi := \{v \in V \mid \phi(v) = 0\} \subseteq V$.

Образ: $\Im \phi := \phi(V) \subseteq W$.

\begin{proposal}~
    \begin{enumerate}[label=(\alph*)]
    \item $\phi$ инъективно $\iff \ker \phi = \{0\}$,
    \item $\phi$ сюръективно $\iff \Im \phi = W$.
    \end{enumerate}
\end{proposal}

\begin{proof}~
    \begin{enumerate}[label=(\alph*)]
    \item 
        \begin{description}[leftmargin=0.935cm]
        \item[$\implies$] очевидно
        \item[$\impliedby$] 
            Пусть $v_1, v_2 \in V$ таковы, что $\phi(v_1) = \phi(v_2)$. 
            Тогда $\phi(v_1 - v_2) = 0$, а значит $v_1 - v_2 \in \ker \phi$.

            Но тогда, $v_1 - v_2 = 0$, то есть $v_1 = v_2$.
        \end{description}
    \item очевидно.
        \qedhere
    \end{enumerate}
\end{proof}

\begin{corollary}
    $\phi$ изоморфизм
    \begin{math}
        \iff
        \left\{
        \begin{aligned}
            \ker \phi &= \{0\},\\
            \Im \phi &= W.
        \end{aligned}
        \right.
    \end{math}
\end{corollary}


\subsection{Характеризация изоморфизмов в терминах их ядер и образов}

Пусть $U \subseteq V$ --- подпространство, $u_1, \dots, u_k$ --- базис в $U$.

\begin{lemma}
    Тогда, $\phi(U) = \left< \phi(u_1), \dots, \phi(u_k) \right>$.
    В частности, $\dim \phi(U) \leq \dim U$ и $\dim \Im \phi \leq \dim V$.
\end{lemma}

\begin{proof}
    $u \in U \implies u = \alpha_1 u_1 + \dots + \alpha_k u_k$, $\alpha_i \in F$, тогда
    \begin{equation*}
        \phi(u) = \alpha_1 \phi(u_1) + \dots + \alpha_k \phi(u_k) \in \left< \phi(u_1), \dots, \phi(u_k) \right>
    .\qedhere\end{equation*}
\end{proof}


\subsection{Связь размерности образа линейного отображения с рангом его матрицы}

Пусть 
\begin{math}
    \begin{aligned}[t]
        \E &= (e_1, \dots, e_n) \text{ --- базис $V$}, \\
        \F &= (f_1, \dots, f_m) \text{ --- базис $W$}, \\
        A &= A(\phi, \E, \F).
    \end{aligned}
\end{math}

\begin{theorem}
    $\rk A = \dim \Im \phi$.
\end{theorem}

\begin{proof}
    По лемме, $\Im \phi = \left< \phi(e_1), \dots, \phi(e_n) \right>$. Поэтому, $\dim \Im \phi = \rk \{\phi(e_1), \dots, \phi(e_n)\}$.

    Так как $j$-й столбец матрицы $A$ составлен из координат вектора $\phi(e_j)$ в базисе $\F$, то
    \begin{equation*}
        \alpha_1 \phi(e_1) + \dots + \alpha_n \phi(e_n) = 0 \iff \alpha_1 A^{(1)} + \dots + \alpha_n A^{(n)} = 0
    .\end{equation*}

    Значит, $\dim \Im \phi = \rk \{\phi(e_1), \dots, \phi(e_n)\} = \rk \{A^{(1)}, \dots, A^{(n)}\} = \rk A$.
\end{proof}

\begin{comment}
    Число $\dim \Im \phi$ называется \textit{рангом} линейного отображения $\phi$, обозначается $\rk \phi$.
\end{comment}

\begin{corollary}
    $\rk A$ не зависит от выбора пары базисов $\E$ и $\F$.
\end{corollary}


\subsection{Инвариантность ранга матрицы относительно умножения на квадратную невырожденную матрицу слева или справа}

\begin{designation}
    $M^{0}_n(F) := \{ C \in M_n(F) \mid \det C \neq 0\}$.  
\end{designation}

\begin{corollary}
    Ранг матрицы не меняется при умножении слева и/или справа на невырожденную матрицу.
\end{corollary}

\begin{proof}
    Если $A \in \text{Mat}_{m \times n}$, $C \in M^{0}_n$, $D \in M_m^{0}$, то $A$ и $D^{-1}AC$ --- это матрицы одного и того же линейного отображения в разных парах базисов.
    По теореме, $\rk A = \rk\left(D^{-1}AC\right)$.
\end{proof}


\subsection{Свойство образов векторов, дополняющих базис ядра до базиса всего пространства}

\begin{proposal}
    Пусть $e_1, \dots, e_k$ --- базис $\ker \phi$ и векторы $e_{k + 1}, \dots, e_n$ дополняют его до базиса всего $V$.

    Тогда, $\phi(e_{k + 1}), \dots, \phi(e_n)$ образуют базис в $\Im \phi$.
\end{proposal}

\begin{proof}
    $\Im \phi = \left< \phi(e_1), \dots, \phi(e_k), \phi(e_{k + 1}), \dots, \phi(e_n) \right> = \left< \phi(e_{k + 1}), \dots, \phi(e_n) \right>$. (так как $\phi(e_1) = \dots = \phi(e_k) = 0$).

    Осталось показать, что $\phi(e_{k + 1}), \dots, \phi(e_n)$ линейно независимы.

    Пусть $\alpha_{k + 1} \phi(e_{k + 1}) + \dots + \alpha_n \phi(e_n) = 0$, где $\alpha_i \in F$.

    Тогда $\phi(\alpha_{k + 1} e_{k + 1} + \dots \alpha_n e_n) = 0 \implies \alpha_{k + 1} e_{k + 1} + \dots + \alpha_n e_n \in \ker \phi$.

    Но тогда $\alpha_{k + 1} e_{k + 1} + \dots \alpha_n e_n = \beta_1 e_1 + \dots + \beta_k e_k$, где $\beta_j \in F$.

    Так как $(e_1, \dots, e_n)$ --- базис $V$, то $\alpha_i = \beta_j = 0 \ \forall i, j$.
\end{proof}


\subsection{Теорема о связи размерностей ядра и образа линейного отображения}

\begin{theorem}
    $\dim \Im \phi + \dim \ker \phi = \dim V$.
\end{theorem}

\begin{proof}
    Вытекает из предыдущего предложения так как в его доказательстве:

    $\dim V = n$,

    $\dim \ker \phi = k$,

    $\dim \Im \phi = n - k$.
\end{proof}


\subsection{Приведение матрицы линейного отображения к диагональному виду с единицами и нулями на диагонали}

\begin{proposal}
    Пусть $\rk \phi = r$. Тогда существует базис $\E$ в $V$ и базис $\F$ в $W$, такие что
    \begin{equation*}
        A(\phi, \E, \F) = \left(
            \begin{array}{c|c}
                E & 0 \\
                \hline
                0 & 0
            \end{array}
        \right) = \bordermatrix{    
            &   & r &   &   &   & n - r &   \cr
            & 1 & 0 & 0 & \dots & 0 & 0 & 0 \cr
          \hspace{0.7cm} r & 0 & \ddots & 0 & \dots & 0 & 0 & 0 \cr
            & 0 & 0 & 1 & \dots & 0 & 0 & 0 \cr
            & \vdots & \vdots & \vdots & \ddots & \vdots & \vdots & \vdots \cr
            & 0 & 0 & 0 & \dots & 0 & 0 & 0 \cr
      m - r & 0 & 0 & 0 & \dots & 0 & 0 & 0 \cr
            & 0 & 0 & 0 & \dots & 0 & 0 & 0
        }
    .\end{equation*}
\end{proposal}

\begin{proof}
    Пусть $e_{r + 1}, \dots, e_n$ --- базис $\ker \phi$. Дополним его векторами $e_1, \dots, e_r$ до базиса всего $V$.

    Положим $f_1 = \phi(e_1)$, $\dots$, $f_r = \phi(e_r)$, тогда $(f_1, \dots, f_r)$ --- базис $\Im \phi$.

    Дополним $f_1, \dots, f_r$ до базиса $f_1, \dots, f_m$ всего $W$.

    Тогда, $\E = (e_1, \dots, e_n)$ и $\F = (f_1, \dots, f_m)$ --- искомые базисы.
\end{proof}

\begin{corollary}
    Если $A \in \text{Mat}_{m \times n}(F)$, $\rk A = r$, то $\exists C \in M_n^{0}(F)$ и $D \in M_m^{0}(F)$, такие что
    \begin{equation*}
        D^{-1}AC = \left(
            \begin{array}{c|c}
                E & 0 \\
                \hline
                0 & 0
            \end{array}
        \right) = B
    .\end{equation*}

    $(\iff A = DBC^{-1})$.
\end{corollary}

\begin{proof}
    Реализуем $A$ как матрицу линейного отображения $\phi \colon F^{n} \to F^m$ в некоторой паре базисов, тогда утверждение вытекает из предложения и формулы изменения матрицы линейного отображения при замене базисов.
\end{proof}


\subsection{Линейные функции на векторном пространстве}

\begin{definition}
    \textit{Линейной функцией} (или \textit{линейной формой}, или \textit{линейным функционалом}) на $V$ называется всякое линейное отображение $\alpha \colon V \to F$.
\end{definition}

\begin{designation}
    $V^{*} := \hom(V, F)$ --- множество всех линейных функций на $V$.
\end{designation}


\subsection{Примеры}

\begin{enumerate}
\item $\alpha \colon F^n \to F$.

    $\alpha \begin{pmatrix} x_1 \\ \dots \\ x_n \end{pmatrix} = (a_1, \dots, a_n) \begin{pmatrix} x_1 \\ \dots \\ x_n \end{pmatrix} = a_1 x_1 + \dots + a_n x_n$, где $a_i \in F$ --- фиксированные скаляры.

\item $F(X, \RR)$ --- все функции из линейного пространства $X$ в $\RR$, $x_0 \in X$,

    $\alpha \colon F(X, \RR) \to \RR$,

    $\alpha(f) := f(x_0)$.

\item $\alpha \colon C[0, 1] \to \RR$

    $\alpha(f) := \displaystyle\int_0^1 f(x) \mathop{}\!d x$

\item $\alpha \colon M_n(F) \to F$

    $\alpha(X) := \tr X$
\end{enumerate}

\subsection{Двойственное (сопряжённое) векторное пространство, его размерность в конечномерном случае}

Из общей теории линейных отображений:
\begin{enumerate}
\item $V^{*}$ --- векторное пространство (оно называется \textit{сопряженным} или \textit{двойственным}).
\item Если $\E = (e_1, \dots, e_n)$ --- фиксированный базис в $V$, то есть изоморфизм $V^{*} \simeq \text{Mat}_{1 \times n}(F)$ (а это ни что иное, как строки длины $n$).

    $\alpha \to (\alpha_1, \dots, \alpha_n)$

    $v = x_1 e_1 + \dots + x_n e_n$

    $\alpha(v) = (\alpha_1, \dots, \alpha_n) \begin{pmatrix} x_1 \\ \dots \\ x_n \end{pmatrix} = \alpha_1 x_1 + \dots + \alpha_n x_n$.

    $\alpha_i = \alpha(e_i)$ --- коэффициенты линейной функции $\alpha$ в базисе $\E$.
\end{enumerate}

\begin{corollary}
    $\dim V^{*} = \dim V$ ($\implies V^{*} \simeq V$).
\end{corollary}


\subsection{Двойственный базис}

При $i = 1, \dots, n$ рассмотрим линейную функцию $\epsilon_i \in V^{*}$, соответствующую строке $(0 \dots 1 \dots 0)$. Тогда $\epsilon_1, \dots, \epsilon_n$ --- базис $V^{*}$, он однозначно определяется условием $\epsilon_i(e_j) = \delta_{ij} = \begin{cases}
    1, &i = j, \\
    0, &i \neq j.
\end{cases}$. ($\delta_{ij}$ --- символ Кронекера)


\begin{definition}
    Базис $(\epsilon_1, \dots, \epsilon_n)$ пространства $V^{*}$, определенный условием выше, называется базисом, \textit{двойственным} (сопряженным) к базису $\E$.

    Удобная запись условия:
    \begin{equation*}
        \begin{pmatrix} \epsilon_1 \\ \dots \\ \epsilon_n \end{pmatrix} (e_1, \dots, e_n) = E
    .\end{equation*}
\end{definition}
