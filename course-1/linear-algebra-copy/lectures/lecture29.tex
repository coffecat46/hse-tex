\section{Лекция 23.04.2020} 

Конспект полностью написан по
\href{https://www.dropbox.com/s/ze7leityir3zbqo/LA_19-20_osn_Lecture29.svg?dl=0}{снимку доски}, 
\href{https://www.youtube.com/watch?v=_J8hatdsSrM}{записи лекции} и
\href{https://www.dropbox.com/s/as7uz9v74ba9u5f/LinOperators2.pdf?dl=0}{слайдам},
возможны баги при переписывании. Если хочется понять точно ли что-то правда, лучше смотреть туда.


\subsection{Критерий диагонализуемости линейного оператора в терминах его характеристического многочлена, а также алгебраической и геометрической кратностей его собственных значений}

Пусть $V$ --- векторное пространство над $F$, $\dim V = n$, $\phi \in \L(V)$ --- линейный оператор.

\begin{theorem}{(критерий диагонализуемости)}
    $\phi$ диагонализуемо $\iff$ выполняются одновременно следующие 2 условия:
    \begin{enumerate}
    \item $\chi_\phi(t)$ разлагается на линейные множители.
    \item $\forall \lambda \in \spec \phi \quad g_\lambda = a_\lambda$.
    \end{enumerate}
\end{theorem}

\begin{proof}~
    \begin{description}
    \item[$\implies$] $\phi$ диагонализуемо $\implies \exists$ базис $\E = (e_1, \dots, e_n)$, такой что $\chi_\phi(t)$ разлагается на линейные множители:
        \begin{equation*}
            A(\phi, \E) = \begin{pmatrix} 
                \mu_1 & 0 & \dots & 0 \\
                0 & \mu_2 & \dots & 0 \\
                \vdots & \vdots & \ddots & \vdots \\
                0 & 0 & \dots & \mu_n
            \end{pmatrix} \implies \chi_\phi(t) = (-1)^n \begin{vmatrix} 
                \mu_1 - t & 0 & \dots & 0 \\
                0 & \mu_2 - t & \dots & 0 \\
                \vdots & \vdots & \ddots & \vdots \\
                0 & 0 & \dots & \mu_n - t
            \end{vmatrix} = (t - \mu_1) \cdot \ldots \cdot (t - \mu_n)
        .\end{equation*}

        Перепишем $\chi_\phi(t)$ в виде $\chi_\phi(t) = (t - \lambda_1)^{k_1} \cdot \ldots \cdot (t - \lambda_s)^{k_s}$, где $ \{\mu_1, \dots, \mu_n\} = \{\lambda_1, \dots, \lambda_s\}, \quad \lambda_i \neq \lambda_j$ при $i \neq j$.

        $\forall i = 1, \dots, s$ имеем $V_{\lambda_i}(\phi) \supseteq \left< e_j \mid \mu_j = \lambda_i \right> \implies \dim V_{\lambda_i}(\phi) \geq k_i$, то есть $g_{\lambda_i} \geq a_{\lambda_i}$.

        Но мы знаем, что $g_{\lambda_i} \leq a_{\lambda_i}$. Следовательно, $a_{\lambda_i} = g_{\lambda_i}$.

    \item[$\impliedby$] Пусть $\chi_\phi(t) = (t - \lambda_1)^{k_1} \cdot \ldots \cdot (t - \lambda_s)^{k_s}$, $\lambda_i \neq \lambda_j$ при $i \neq j$.

        Так как подпространства $V_{\lambda_1}(\phi), \dots, V_{\lambda_s}(\phi)$ линейно независимы, то 
        \begin{equation*}
            \dim(V_{\lambda_1}(\phi) + \dots + V_{\lambda_s}(\phi)) = \dim V_{\lambda_1}(\phi) + \dots + \dim V_{\lambda_s}(\phi) = k_1 + \dots + k_s = n = \dim V
        .\end{equation*}
        Следовательно, $V = V_{\lambda_1}(\phi) \oplus \dots \oplus V_{\lambda_s}(\phi)$.

        Если $\E_i$ --- базис в $V_{\lambda_i}(\phi)$, то $\E = \E_1 \sqcup \dots \sqcup \E_s$ --- базис всего $V$, состоящий из собственных векторов, а значит $\phi$ диагонализуем.
        \qedhere
    \end{description}
\end{proof}

\paragraph{Примеры}

\begin{enumerate}
    \item $\phi = \lambda \cdot \mathrm{Id}$ --- скалярный оператор.

        Для всякого базиса $\E$ в $V$ имеем $A(\phi, \E) = \diag(\lambda, \dots, \lambda)$.

        $\chi_\phi(t) = (t - \lambda)^n$.

        $\spec \phi = \{\lambda\}$, $a_\lambda = n = g_\lambda \implies $ условия 1) и 2) выполнены.

    \item $V = \RR^2$, $\phi$ --- ортогональная проекция на прямую $l \ni 0$.

        $e_1 \in l \setminus \{0\}$, $e_2 \in l^{\perp} \setminus \{0\}$, $\E = (e_1, e_2) \implies A(\phi, \E) = \begin{pmatrix} 1 & 0 \\ 0 & 0 \end{pmatrix}$

        $\chi_\phi(t) = t(t - 1) \implies \spec \phi = \{0, 1\}$.

        $\lambda = 0, 1 \implies a_\lambda = 1 = g_\lambda \implies $ условия 1) и 2) выполнены.

    \item $V = \RR^2$, $\phi$ --- поворот на угол $\alpha \neq \pi k$.

        $\E = (e_1, e_2)$ --- положительно ориентированный базис $\implies A(\phi, \E) = \begin{pmatrix} \cos \alpha & - \sin \alpha \\ \sin \alpha & \cos \alpha \end{pmatrix}$.

        $\chi_\phi(t) = \begin{vmatrix} \cos \alpha - t & - \sin \alpha \\ \sin a & \cos \alpha - t \end{vmatrix} = t^2 - 2 \cos \alpha \cdot t + 1$.

        $\frac{D}{4} = \cos ^2 \alpha - 1 = - \sin ^2 a < 0 \implies $ нет корней в $\RR \implies \chi_\phi(t)$ не разлагается на линейные множители над $\RR \implies $ 1) не выполнено $ \implies \phi$ не диагонализуем над $\RR$.

        Однако $\phi$ диагонализуем над $\CC$!

    \item $V = F[x]_{\leq n}$, $n \geq 1$; $\quad \phi \colon f \mapsto f'$.

        Техническое условие: $\mathop{\mathrm{char}} F = 0$ ($\iff \mathop{\mathrm{ord}} 1 = \infty$ в группе $(F, +)$), например, $F = \QQ, \RR, \CC$ подходят.

        \begin{math}
            \E = (1, x, \dots, x^n) \implies A(\phi, \E) = \begin{pmatrix}
                0 & 1 & 0 & \dots & 0 \\
                0 & 0 & 2 & \dots & 0 \\
                \vdots & \vdots & \vdots & \ddots & \vdots \\
                0 & 0 & 0 & \dots & n \\
                0 & 0 & 0 & \dots & 0
            \end{pmatrix}
        \end{math}

        $\chi_\phi(t) = t^{n + 1} \implies \spec \phi = \{0\} \implies $ 1) выполнено.

        $\lambda = 0 \implies a_\lambda = n + 1; \quad V_\lambda(\phi) = \left< 1 \right> \implies g_\lambda = 1 < a_\lambda \implies $ 2) не выполнено $\implies \phi$ не диагонализуем.

        \begin{math}
            \E' = \left(1, x, \frac{x^2}{2}, \dots, \frac{x^n}{n!}\right) \implies A(\phi, \E') = \begin{pmatrix} 
                0 & 1 & 0 & \dots & 0 \\
                0 & 0 & 0 & \dots & 0 \\
                \vdots & \vdots & \vdots & \ddots & \vdots \\
                0 & 0 & 0 & \dots & 1 \\
                0 & 0 & 0 & \dots & 0
            \end{pmatrix} = J_0^{n + 1} \text{ --- это жорданова клетка}
        \end{math}
\end{enumerate}


\subsection{Существование одномерного или двумерного инвариантного подпространства у линейного оператора в действительном векторном пространстве}

\begin{theorem}
    $F = \RR \implies \forall \phi \in L(V) \ \exists $ либо $1$-мерное, либо $2$-мерное $\phi$-инвариантное подпространство.
\end{theorem}

\begin{proof}
    Если $\chi_\phi(t)$ имеет действительные корни, то в $V$ есть собственный вектор $ \implies $ 1-мерное $\phi$-инвариантное подпространство.

    Пусть $\chi_\phi(t)$ не имеет корней в $\RR$. Возьмем какой-нибудь комплексный корень $\lambda + i \mu$, $\mu \neq 0$.

    Фиксируем базис $\E$ в $V$ и положим $A = A(\phi, \E)$. Для $\lambda + i \mu$ у матрицы $A$ существует комплексный собственный вектор, то есть такое $u, v \in \RR^n$, что
    \begin{equation*}
        A(u + iv) = (\lambda + i \mu) (u + i v) \implies Au + iAv = \lambda u - \mu v + i (\lambda v + \mu u) \implies \begin{cases}
            Au &= \lambda u - \mu v \\
            Av &= \lambda v + \mu u
        \end{cases}
    .\end{equation*}
    
    Значит, векторы в $V$ с координатами $u, v$ порождают $\phi$-инвариантное подпространство $U \subseteq V$ размерности $ \leq 2$.
\end{proof}

\begin{exercise}
    $\dim U = 2$.
\end{exercise}


\subsection{Отображение, сопряжённое к линейному отображению между двумя евклидовыми пространствами: определение, существование и единственность. Матрица сопряжённого отображения в паре произвольных и паре ортонормированных базисов}

Пусть 
\begin{math}
    \begin{aligned}[t]
        &\EE \text{ --- евклидово пространство со скалярным произведением } (\bigcdot, \bigcdot), \quad \dim \EE = n, \\
        &\EE \text{ --- другое евклидово пространство со скалярным произведением } (\bigcdot, \bigcdot)', \quad \dim \EE' = m, \\
        &\phi \colon \EE \to \EE'.
    \end{aligned}
\end{math}

\begin{definition}
    Линейное отображение $\psi \colon \EE' \to \EE$ называется \textit{сопряженным} к $\phi$, если 
    \begin{equation*}
        \label{lec29:def}
        \tag{$\star$}
        (\phi(x), y)'  = (x, \psi(y)) \quad \forall x \in \EE, y \in \EE'
    .\end{equation*}

    Обозначение: $\phi^*$.
\end{definition}

\begin{proposal}~
    \begin{enumerate}
    \item $\psi$ существует и единственно.
    \item Если $\E$ --- базис $\EE$, $\F$ --- базис $\EE'$, 
        \begin{math}
            \begin{aligned}
                G &= G(e_1, \dots, e_n) \\
                G' &= G(f_1, \dots, f_m)
            \end{aligned}
        \end{math} и 
        \begin{math}
            \begin{aligned}
                A_\phi &= A(\phi, \E, \F) \\
                A_\psi &= (A, \psi, \F, \E),
            \end{aligned}
        \end{math} то $A_\psi = G^{-1} A_\phi^{T} G'$.

        В частности, если $\E$ и $\F$ ортонормированы, то $A_\psi = A_\phi^{T}$.
    \end{enumerate}
\end{proposal}

\begin{proof}
    $x = x_1 e_1 + \dots + x_n e_n \in \EE$, $y = y_1 f_1 + \dots + y_m f_m \in \EE'$.

    \begin{equation*}
        (\phi(x), y)' = \left(A_\phi \begin{pmatrix} x_1 \\ \dots \\ x_n \end{pmatrix} \right)^{T} \cdot G' \cdot \begin{pmatrix} y_1 \\ \dots \\ y_m \end{pmatrix} = (x_1 \dots x_n) \cdot A_\phi^{T} \cdot G' \cdot \begin{pmatrix} y_1 \\ \dots \\ y_m \end{pmatrix}
    .\end{equation*}

    \begin{equation*}
        (x, \psi(y)) = (x_1 \dots x_n) \cdot G \cdot A_\psi \cdot \begin{pmatrix} y_1 \\ \dots \\ y_m \end{pmatrix}
    .\end{equation*}

    Так как $\forall B \in \text{Mat}_{m \times n} \quad b_{ij} = \underset{i}{(0 \dots 0 \ 1 \ 0 \dots 0)} \cdot B \cdot \underset{j}{(0 \dots 0 \ 1 \ 0 \dots 0)}^{T}$, то ${\eqref{lec29:def} \iff A_{\phi}^{T} G' = G A_\psi \iff A_\psi = G^{-1} A^{T}_\phi G'}$.

    Отсюда следуют сразу оба утверждения.
\end{proof}


\subsection{Сопряжённый оператор в евклидовом пространстве}

\subsection{Самосопряжённые (симметрические) операторы}

Пусть теперь $\EE' = \EE$.

$\phi \colon \EE \to \EE$ --- линейный оператор $ \implies \exists!$ линейный оператор $\phi^{*} \colon \EE \to \EE$, такой что $(\phi(x), y) = (x, \phi^{*}(y)) \quad \forall x, y \in \EE$.

\begin{definition}
    Линейный оператор $\phi \in L(\EE)$ называется \textit{самосопряженным} (или \textit{симметричным}), если $\phi = \phi^{*}$, то есть $(\phi(x), y) = (x, \phi(y)) \quad \forall x, y \in \EE`$.
\end{definition}


\subsection{Существование собственного вектора у самосопряжённого оператора}

Если $\E$ --- ортонормированный базис в $\EE$, $A_\phi = A(\phi, \E)$, $A_{\phi^{*}} = A(\phi^{*}, \E)$, то $A_{\phi^{*}} = A_\phi^{T}$.

Следовательно, $\phi = \phi^{*} \iff A_\phi = A_\phi^{T}$.

\begin{proposal}
    Если $\phi = \phi^{*}$, то $\exists$ собственный вектор для $\phi$.
\end{proposal}

\begin{proof}
    Было: $\exists$ либо
    \begin{enumerate*}[label=\arabic*)]
    \item 1-мерное $\phi$-инвариантное подпространство, либо
    \item 2-мерное $\phi$-инвариантное подпространство.
    \end{enumerate*}

    \begin{enumerate}
    \item ок.
    \item $U \subseteq \EE$ --- $\phi$-инвариантное подпространство, $\dim U = 2$.

        Фиксируем ортонормированный базис $\E = (e_1, e_2)$. Пусть $\psi = \phi \big|_{U}$.

        Значит, $\psi = \psi^{*} \implies A(\psi, \E) = \begin{pmatrix} a & b \\ b & c \end{pmatrix}$.

        Отсюда, $\chi_\psi(t) = \begin{vmatrix} a - t & b \\ b & c - t \end{vmatrix} = t^2 - (a + c)t + ac - b^2$.

        $D = (a + c)^2 - 4(ac - b^2) = (a - c)^2 + 4b^2 \geq 0$.

        Следовательно, $\chi_\psi(t)$ имеет корни в $\RR$, то есть в $U$ есть собственный вектор для $\psi$, он же собственный вектор для $\phi$.
        \qedhere
    \end{enumerate}
\end{proof}


\subsection{Инвариантность ортогонального дополнения к подпространству, инвариантному относительно самосопряжённого оператора}

\begin{proposal}
    $\phi = \phi^{*}$, $U \subseteq \EE$ --- $\phi$-инвариантное подпространство, тогда $U^{\perp}$ --- тоже $\phi$-инвариантное подпространство.
\end{proposal}

\begin{proof}
    $\phi(U) \subseteq U$, хотим $\phi(U^{\perp}) \subseteq U^{\perp}$. $\quad \forall x \in U^{\perp} \quad \forall y \in U \quad (\phi(x), y) = (x, \phi(y)) = 0 \implies \phi(x) \in U^{\perp}$.
\end{proof}
