\section{Лекция 10}

\subsection{Векторные пространства, простейшие следствия из аксиом}

\subsubsection{Определение векторного пространства}

Фиксируем поле $F$ (можно считать, что $F = \RR$ или $\CC$)

\begin{definition}
    Множество $V$ называется \textit{векторным (линейным) пространством} над полем $F$, если на $V$ заданы две операции
    \begin{itemize}[nosep]
    \item ``сложение'': $V \times V \to V$, $(x, y) \mapsto x + y$.
    \item ``умножение на скаляр'': $F \times V \to V$, $(\alpha \in F, x \in V) \mapsto \alpha x$.
    \end{itemize}
    а также, $\forall x, y, z \in V$ и $\alpha, \beta \in F$ выполнены следующие условия (называются \textit{аксиомами векторного пространства}):
    \begin{enumerate}[nosep]
    \item $x + y = y + x$.
    \item $(x + y) + z = x + (y + z)$.
    \item $\exists \overrightarrow{0} \in V : x + \overrightarrow{0} = \overrightarrow{0} + x = x$ (нулевой элемент).
    \item $\exists -x : -x + x = x + (-x) = \overrightarrow{0}$ (противоположный элемент).
    \item $\alpha(x + y) = \alpha x + \alpha y$.
    \item $(\alpha + \beta)x = \alpha x + \beta x$.
    \item $(\alpha \beta)x = \alpha(\beta x)$.
    \item $1 \cdot x = x$.
    \end{enumerate}
\end{definition}

\begin{definition}
    Элементы векторного пространства называются (абстрактными) \textit{векторами}.
\end{definition}

\begin{example}
    \begin{enumerate}~
    \item $\RR$ над $\RR$ (или $F$ над $F$).
    \item Пространство $\RR^n$ над $\RR$ (или $F^n$ над $F$) реализованное как пространство столбцов или строк длины $n$.
    \item $\text{Mat}_{m \times n}(F)$.
    \item $F[x]$ -- многочлены то переменной $x$ с коэффициентами в $\RR$.
    \item Пространство функций на множестве $M$ с значениями в $F$:

        $f\colon M \rightarrow \RR$

        \begin{itemize}[nosep]
        \item сложение $(f_1 + f_2)(x) := f_1(x) + f_2(x)$.
        \item умножение на скаляр $(\alpha f)(x) := \alpha f(x)$.
        \end{itemize}

        -- это векторное пространство над $F$.

        Например, множество всех функций $[0, 1] \to R$.
    \end{enumerate}
\end{example}


\subsubsection{Простейшие следствия из аксиом}

$\forall \alpha \in F, x \in V$.
\begin{enumerate}
\item Элемент $\overrightarrow{0}$ единственный.

    Если $\overrightarrow{0}'$ -- другой такой ноль, то $\overrightarrow{0}' = \overrightarrow{0}' + \overrightarrow{0} = \overrightarrow{0}$.

\item Элемент $-x$ единственный.

    Если $(-x)'$ -- другой такой противоположный элемент, то
    \begin{equation*}
        (-x)' = (-x)' + \overrightarrow{0} = (-x)' + (x + (-x)) = ((-x)' + x) + (-x) = \overrightarrow{0} + (-x) = -x
    .\end{equation*}

\item $\alpha \overrightarrow{0} = \overrightarrow{0}$.

    \newcommand{\zr}{\overrightarrow{0}}
    Рассмотрим равенство $\zr + \zr = \zr$. Домножив на $\alpha$ получаем $\alpha(\zr + \zr) = \alpha\zr$.

    Раскроем скобки, $\alpha\zr + \alpha\zr = \alpha\zr$.

    Прибавим к обоим частям обратный элемент к $\alpha\zr$, получим $\alpha\zr + \zr = \zr \implies \alpha\zr = \zr$.

\item $\alpha (-x) = -(\alpha x)$.

    Рассмотрим равенство $x + (-x) = \overrightarrow{0}$.
    \begin{align*}
        x + (-x) = \overrightarrow{0} \implies ax + a(-x) = 0 \implies a(-x) = -(ax)
    .\end{align*}

\item $0 \cdot x = \overrightarrow{0}$.

    Доказывается так же, как пункт 3, но с 0 вместо $\zr$.

\item $(-1) \cdot x = -x$.

    Рассмотрим равенство $1 + (-1) = 0$. Домножив на $x$ получаем $(1 + (-1))x = 0x$.

    Раскроем скобки и воспользуемся пунктом 5 --- $1x + (-1)x = 0$ или $x + (-1)x = 0$.

    Прибавим к обоим частям $-x$, получим $0 + (-1)x = -x$ или $(-1)x = -x$.

\end{enumerate}


\subsection{Подпространства векторных пространств}


Пусть $V$ -- векторное пространство над $F$.

\begin{definition}
    Подмножество $U \subseteq V$ называется \textit{подпространством} (в $V$), если
    \begin{enumerate}[nosep]
    \item $\overrightarrow{0} \in U$.
    \item $x, y \in U \implies x + y \in U$.
    \item $x \in U, \alpha \in F \implies \alpha x \in U$.
    \end{enumerate}
\end{definition}

\begin{comment}
    Всякое подпространство само является векторным пространством относительно тех же операций.
\end{comment}

\begin{example}~
    \begin{enumerate}
        \item $\{\overrightarrow{0}\}$ и $V$ -- всегда подпространства в $V$.

            они называются \textit{несобственными} подпространствами, остальные называются \textit{собственными}.

        \item Множество всех верхнетреугольных, нижнетреугольных, диагональных матриц в $M_n(F)$.

        \item $F[x]_{\leq n}$ все многочлены в $F[x]$ степени $\leq n$ -- подпространство в $F[x]$.
    \end{enumerate}
\end{example}


\subsection{Утверждение о том, что множество решений однородной системы линейных уравнений с n неизвестными является подпространством в $F^n$}

\begin{proposal}
    Множество решений любой ОСЛУ $Ax = 0$ ($A \in \text{Mat}_{m \times n}(F)$, $x \in F^n$) является подпространством в $F^n$.
\end{proposal}

\begin{proof}
    Пусть $S$ -- множество решений ОСЛУ Ax = 0.
    \begin{enumerate}
        \item $\overrightarrow{0} = \begin{pmatrix} 0 \\ \vdots \\ 0 \end{pmatrix} \in S$.
        \item $x, y \in S \implies Ax = \overrightarrow{0}$ и $Ay = \overrightarrow{0} \implies A(x + y) = Ax + Ay = \overrightarrow{0} + \overrightarrow{0} = \overrightarrow{0} \implies x + y \in S$.
        \item $x \in S, \alpha \in F \implies Ax = \overrightarrow{0} \implies A(\alpha x) = \alpha (Ax) = \alpha \overrightarrow{0} = \overrightarrow{0} \implies \alpha x \in S$. \qedhere
    \end{enumerate}
\end{proof}


\subsection{Линейная комбинация конечного набора векторов}

Пусть $V$ -- векторное пространство над $F$ и $v_1, \dots, v_k \in V$ -- набор векторов.

\begin{definition}
    \textit{Линейной комбинацией} векторов $v_1, \dots, v_k$ называется всякое выражение вида $\alpha_1 v_1 + \dots + \alpha_k v_k$, где $\alpha_i \in F$.
\end{definition}

\subsection{Линейная оболочка подмножества векторного пространства, примеры}

Пусть $S \subseteq V$ -- подмножество векторного пространства.

\begin{definition}
    \textit{Линейной оболочкой} множества $S$ называются множество всех векторов из $V$, представимых в виде линейной комбинации какого-то конечного набора векторов из $S$.

    Обозначение: $\langle S \rangle$.
\end{definition}

Если $S = \{v_1, \dots, v_k\}$ конечно и состоит из векторов $v_1, \dots, v_k$, то еще пишут $\langle v_1, \dots, v_k \rangle$ и говорят ``линейная оболочка векторов $v_1, \dots, v_k$''.

Соглашение: $\langle \varnothing \rangle = \{\overrightarrow{0}\}$.

\begin{example}~
    \begin{enumerate}
    \item
        $\langle \overrightarrow{0} \rangle = \{\overrightarrow{0}\}$.
    \item
        $V = \RR^2$, $v \neq 0$, $\langle v \rangle = \{\alpha v \mid \alpha \in \RR\}$ -- прямая.
    \item
        $V = \RR^3$, $v_1, v_2$ -- пара неколлинеарных векторов.

        Тогда, $\langle v_1, v_2 \rangle = \{a_1 v_1 + a_2 v_2 \mid a_1, a_2 \in \RR\}$ -- плоскость натянутая на $v_1, v_2$.
    \end{enumerate}
\end{example}
