% Здесь НЕ НУЖНО делать begin document, включать какие-то пакеты..
% Все уже подрубается в головном файле
% Хедер обыкновенный хсе-теха, все его команды будут здесь работать
% Пожалуйста, проверяйте корректность теха перед пушем

% Здесь формулировка билета
\subsection{Сформулируйте определение интеграла Римана от функции по жордановому множеству. Что такое интегрируемая функция?}

Пусть $f: D \rightarrow \mathbb{R}$ - заданная на $D$ числовая функция и $\tau = \{D_i\}$ - разбиение множества $D$. Выберем произвольно точки $\xi_i \in D_i$. Систему выбранных точек будем обозначать $p = \{\xi_i\}$ \\
\textbf{\underline{Опр.:} } \textit{Интегральной суммой} функции $f$ на жордановом множестве $D$, соответствующей разбиению $\tau$ и выбору точек $p$, называется 
\[I_D(f, \tau, p) = \sum\limits_if(\xi_i)\mu(D_i)\]
\textbf{\underline{Опр.:} } Функция $f$ называется \textit{интегрируемой по Риману} на $D$, если существует такое число $I$, что
\[\forall \varepsilon > 0 \ \exists \delta > 0 \ \ |I_D(f, \tau, p) - I| < \varepsilon \ \text{при} \ \Delta(\tau) < \delta\]
Причем это число $I$ называется \textit{интегралом Римана} функции $f$ на $D$ и обозначается
\[I = \int\limits_Df(x)dx\]
Множество всех функций, интегрируемых по Риману на жордановом множестве $D$, обозначается $\mathcal{R}(D)$

