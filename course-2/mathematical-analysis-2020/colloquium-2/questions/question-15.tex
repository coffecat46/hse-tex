% Здесь НЕ НУЖНО делать begin document, включать какие-то пакеты..
% Все уже подрубается в головном файле
% Хедер обыкновенный хсе-теха, все его команды будут здесь работать
% Пожалуйста, проверяйте корректность теха перед пушем

% Здесь формулировка билета
\subsection{Сформулируйте и докажите критерий Коши интегрируемости функции по Риману.}

\textbf{\underline{Теор.:} } Пусть $f$ - некоторая функция. Если для любого $\varepsilon > 0$ найдется $\delta > 0$, что при любом выборе разбиений $\tau, \tau'$, с диаметрами $\Delta(\tau), \Delta(\tau') < \delta$ и при любом выборе систем точек $p, p'$ выполняется 
\[|I_D(f, \tau, p) - I_D(f, \tau', p')| < \varepsilon,\]
то функция интегрируема по Риману.\\
***ПРОВЕРИТЬ ДОКАЗАТЕЛЬСТВО*** \\
\textbf{\underline{Док-во:} } \\
$\Leftarrow$ Пусть $f \in \mathcal{R}(D)$, тогда существует $I$, такое что 
\[|I_D(f, \tau, p) - I| < \frac{\varepsilon}{2}\]
\[|I_D(f, \tau', p') - I| < \frac{\varepsilon}{2}\]
отсюда имеем
\[|I_D(f, \tau, p) - I_D(f, \tau', p')| \leq |I_D(f, \tau, p) - I| + |I_D(f, \tau', p') - I| < \varepsilon\]
$\Rightarrow$ Возьмем последовательности $\tau_n$ и $p_n$, причем $\Delta(\tau_n) \rightarrow 0$ \\
С помощью данных последовательностей образуем последовательность интегральных сумм $I_D(f, \tau_n, p_n)$. \\
Теперь положим, что выполнен критерий Коши
\[|I_D(f, \tau_m, p_m) - I_D(f, \tau_n, p_n)| < \varepsilon\]
из чего следует, что $I_D(f, \tau_n, p_n) \rightarrow I$, где $I$ - некоторое число. \\
Теперь в исходное неравенство подставим $I_D(f, \tau_n, p_n)$ и устримим его к $I$
\[|I_D(f, \tau, p) - I_D(f, \tau_n, p_n)| < \varepsilon\]
\[|I_D(f, \tau, p) - I| < \varepsilon\] 
из чего следует, что функция интегрируема по Риману.
\begin{flushright}
$\blacksquare$
\end{flushright}


