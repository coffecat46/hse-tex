% Здесь НЕ НУЖНО делать begin document, включать какие-то пакеты..
% Все уже подрубается в головном файле
% Хедер обыкновенный хсе-теха, все его команды будут здесь работать
% Пожалуйста, проверяйте корректность теха перед пушем

% Здесь формулировка билета
\subsection{Что такое множество лебеговой меры нуль? В каком случае функция называется непрерывной почти всюду на множестве?}

\textbf{\underline{Опр.:} } Множество $A \subset \mathbb{R}^m$ имеет $m$-мерную \textit{меру Лебега нуль}, если для любого $\varepsilon > 0$ существует счетный набор $m$-мерных полуинтервалов
\[Q_i = [a_i^1; b_i^1) \times ... \times [a_i^m; b_i^m), \ \ \ \ i \in \mathbb{N}\]
имеющий сумму мер 
\[\sum\limits_{i=1}^{\infty}\mu(Q_i) < \varepsilon\]
и объединение которых покрывает $A$
\[A \subseteq \bigcup_{i\in\mathbb{N}}Q_i\]
\textbf{\underline{Опр.:} } Функция $f$, определенная на множестве $D$, называется непрерывной на $D$ \textit{почти всюду}, если существует такое множество $A$ лебеговой меры нуль, что $f$ непрерывна на $D\backslash A$


