% Здесь НЕ НУЖНО делать begin document, включать какие-то пакеты..
% Все уже подрубается в головном файле
% Хедер обыкновенный хсе-теха, все его команды будут здесь работать
% Пожалуйста, проверяйте корректность теха перед пушем

% Здесь формулировка билета
\subsection{Докажите, что образ жорданова множества при диффеоморфизме является жордановым множеством.}

Если $A$~--- открытое множество, то и $B = \varphi(A)$ открыто.

Если $A$~--- замкнутое множество, то и $B = \varphi(A)$ замкнуто.

\begin{theorem*}
    Если $A$~--- жорданово множество, то и $B = \varphi(A)$~--- жорданово множество.
\end{theorem*}
\begin{proof}
    Что такое жорданово множество? Это множество, множество граничных точек которого имеет сколь угодно малую внешнюю жорданову меру (имеет жорданову меру нуль). То есть, можно найти простое множество сколь угодно малой меры, которое покрывает границу множества $A$. Множество $A$ жорданово $\iff$ его граница покрывается простым множеством сколь угодно малой меры.
    
    Как мы знаем, граница переходит в границу, то есть граница множества $B$~--- образ границы множества $A$ при отображении $\varphi$. Поскольку отображение $\varphi$ непрерывно, а граница любого множества является замкнутым множеством, на границе множества $A$ функция $\varphi$ будет равномерно непрерывной, следовательно, различие в образах при отображении $\varphi$ будет меньше $\varepsilon$ если только различие между их прообразами меньше $\delta$.
    
    Для любого $\varepsilon$ я смогу указать такое $\delta$, что если расстояние между точками границы множества $A$ меньше $\delta$, то и расстояние между соответствующими точками границы множества $B$ будет меньше $\varepsilon$, следовательно, если граница множества $A$ покрыта полуинтервалами длины меньше $\delta$, то граница множества $B$ покрывается полуинтервалами длины меньше $\varepsilon$~--- это следствие равномерной непрерывности отображения $\varphi$.
    
    А если граница множества $B$ покрывается интервалами длины меньше $\varepsilon$, то всё множество $B$ будет покрываться конечным числом полуинтервалов длины меньше $\varepsilon$, то есть будет иметь сколь угодно малую меру. Тем самым, граница множества $B$ имеет сколь угодно малую внешнюю меру Жордана, то есть меру Жордана, равную нулю. Это и означает, что множество $B$~--- множество, измеримое по Жордану, или жорданово.

    
\end{proof}