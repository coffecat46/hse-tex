% Здесь НЕ НУЖНО делать begin document, включать какие-то пакеты..
% Все уже подрубается в головном файле
% Хедер обыкновенный хсе-теха, все его команды будут здесь работать
% Пожалуйста, проверяйте корректность теха перед пушем

% Здесь формулировка билета
\subsection{Выведите формулу для меры жорданова множества в криволинейных координатах}

\begin{proposition*}
    Так как плотность меры $\nu$, равная $|J_\varphi(u)|$, непрерывна, то
    \begin{equation*}
        \nu(A) = \int_A |J_\varphi(u)| du
    \end{equation*}    
\end{proposition*}
\begin{proof}
    Рассмотрим разбиение множества $G = \bigsqcup G_i$, $D_i = \varphi(G_i)$, $D = \bigsqcup D_i$. 

    \begin{equation*}
        \mu(D_i) = \mu(\varphi(G_i)) = \nu(G_i) = \int_{G_i} |J_\varphi(u)| du
    \end{equation*}
\end{proof}
