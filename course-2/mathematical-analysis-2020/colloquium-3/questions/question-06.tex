\subsection{Свойства равномерно сходящегося несобственного интеграла. 
Теорема о дифференцировании по параметру под знаком несобственного интеграла. 
Теорема о собственном интегрировании по параметру под знаком несобственного интеграла. 
Теорема о несобственном интегрировании по параметру под знаком несобственного интеграла.}

\subsubsection{ Теорема о дифференцировании по параметру под знаком несобственного интеграла.}
\[ F(y) = \int_a^{\omega} f(x, y)\,dx, \ \ \ y \in [c; d] \]
Пусть: $\ f(x, y), \; f_y'(x, y)$ --- непрерывны на $[a; \omega) \times [c; d]$ \\
\phantom{Пусть} $\Phi(y) = \int_a^{\omega} f_y'(x, y)\,dx$ --- сходится равномерно на $[c; d]$ \\
\phantom{Пусть} $\int_a^{\omega} f(x, y)\,dx$ сходится хотя бы в 1 точке $y_0 \in [c; d]$

Тогда:
\[ \int_a^{\omega} f(x, y)\,dx \text{ сходится равномерно на } [c; d], \]
причем $F(y)$ дифференцируема на $[c; d]$ и $F'(y) = \Phi(y)$
\begin{proof}
    \[ g(t, y) = \int_a^t f(x, y)\,dx \]
    По теореме о дифференцировании собственного интеграла по параметру, $g(t, y)$ дифференцируема по $y$ на $[c; d]$ и 
    \[ g_y'(t, y) = \int_a^t f_y'(x, y)\,dx \overset{y \in [c; d]}{\underset{t \to \omega}{\rightrightarrows}} \Phi(y) \]
    По условию $\ g(t, y_0) \xrightarrow[t \to \omega]{} F(y_0)$
    
    Рассмотрим семейство $g(t, y)$. По теореме о дифференцировании семейства функций по параметру:
    \[ g(t, y) \overset{y \in [c; d]}{\underset{t \to \omega}{\rightrightarrows}} F(y), \ \ F'(y) = \Phi(y) \]
\end{proof}

\begin{example}
    Вычислим интеграл Дирихле: $\int\limits_0^{+\infty} \frac{\sin x}x\,dx$
    
    Рассмотрим вспомогательный интеграл: $F(y) = \int\limits_0^{+\infty} \frac{\sin x}x \cdot e^{-xy}\,dx, \ y > 0$
    \[ f(x, y) = \frac{\sin x}x \cdot e^{-xy}, \ \ f_y'(x, y) = -\sin x \cdot e^{-xy} 
    \text{ --- непрерывна на } [0; +\infty) \times [c; d], \]
    где $[c; d] \subset (0; +\infty)$
    \[ \Phi(y) = -\int\limits_0^{+\infty} \sin x \cdot e^{-xy}\,dx \text{ --- вспомогательный интеграл} \]
    \[ \left| \sin x \cdot e^{-xy} \right| \le e^{-xy}, \ xy \ge cx, \ \int\limits_0^{+\infty} e^{-cx}\,dx \text{ --- сходится} \]
    \[ \Rightarrow\; \int\limits_0^{+\infty} \sin x \cdot e^{-xy}\,dx \text{ --- равномерно сходится по признаку Вейерштрасса} \]
    При $y_0 > 0 \ \ \int\limits_0^{+\infty} \frac{\sin x}x \cdot e^{-xy}\,dx$ сходится по признаку Абеля \\
    (т.к. $\int\limits_0^{+\infty} \frac{\sin x}x$ --- сходится, а $e^{-xy}$ --- монотонная и ограниченная)
    
    $\Rightarrow$ по теореме можно внести $\frac{d}{dy}$ под знак интеграла:
    \[ F'(y) = -\int\limits_0^{+\infty} \sin x \cdot e^{-xy}\,dx = \star \]
    
    \[ \int \sin x \cdot e^{-xy}\,dx = -\int e^{-xy}\,d(\cos x) = -e^{-xy} \cos x - y \int \cos x \cdot e^{-xy}\,dx = \]
    \[ = -e^{-xy} \cos x - y \int e^{-xy}\,d(\sin x) = -e^{-xy} \cos x - y e^{-xy} \sin x - y^2 \int \sin x \cdot e^{-xy}\,dx \]
    \[ \Rightarrow;\ \int \sin x \cdot e^{-xy}\,dx = -\frac{e^{-xy} (\cos x + y \sin x)}{1 + y^2} + C \ \ (y > 0) \]
    \[ \star = -\left( 0 + \frac1{1 + y^2} \right) \]
    Т.е. $F'(y) = -\frac1{1 + y^2} \;\Rightarrow\; F(y) = -\arctg y + C$
    
    Вычислим $\lim_{y \to +\infty} F(y) = \lim_{y \to +\infty} \int\limits_0^{+\infty} \frac{\sin x}x \cdot e^{-xy}\,dx$
    \[ \left\{\begin{array}{l} 
    \int\limits_0^{+\infty} \frac{\sin x}x \cdot e^{-xy}\,dx \text{ --- сходится равномерно}, \\
    \lim_{y \to +\infty} \frac{\sin x}x \cdot e^{-xy} = 0 \ \ \text{ при } x \in [a; b] \subset (0; +\infty), \\
    \text{причем } \ \frac{\sin x}x \cdot e^{-xy} \overset{x \in [a; b]}{\underset{y \to +\infty}{\rightrightarrows}} 0
    \end{array}\right. \]
    \[ \Rightarrow\; \text{можно внести } \lim_{y \to +\infty} \text{ под знак интеграла} \]
    
    \[ \lim_{y \to +\infty} F(y) = \int\limits_0^{+\infty} 0\,dx = 0 \;\Rightarrow\; C = \frac{\pi}2 \]
    Итак: $F(y) = -\arctg y + \frac{\pi}2$
    Осталось внести $\lim_{y \to +\infty}$ под интеграл $F(y)$ и получить интеграл Дирихле
    \[ \left\{\begin{array}{l} 
    \int\limits_0^{+\infty} \frac{\sin x}x\,e^{-xy}\,dx \text{ --- сходится равномерно по признаку Абеля}, \\
    \lim_{y \to +0} \frac{\sin x}x\,e^{-xy} = \frac{\sin x}x, \\
    \left| \frac{\sin x}x\,e^{-xy} - \frac{\sin x}x \right| \le \frac{|\sin x|}x \cdot \left( 1 - e^{-xy} \right) \le 
    \frac1a \cdot \left( 1 - e^{-by} \right) \\
    \Rightarrow\; \frac{\sin x}x\,e^{-xy} \rightrightarrows \frac{\sin x}x
    \end{array}\right. \ \ \ x \in [a; b] \subset (0; +\infty) \]
    \[ \Rightarrow\; \lim_{y \to +0} F(y) = \int\limits_0^{+\infty} \frac{\sin x}x\,dx \]
    \[ \Rightarrow\; \int\limits_0^{+\infty} \frac{\sin x}x\,dx = 
    \lim_{y \to +0} \left( -\arctg y + \frac{\pi}2 \right) = \frac{\pi}2 \]
\end{example}

\subsubsection{ Теорема о собственном интегрировании по параметру под знаком несобственного интеграла.}
\label{subsubsec:6.2}
\[ \int\limits_c^d F(y)\,dy = \int\limits_c^d dy \int\limits_a^{\omega} f(x, y)\,dx \]
Пусть \\
\phantom{Пусть} $f(x, y)$ непрерывна на $[a, \omega) \times [c, d]$ \\
\phantom{Пусть} $F(y) = \int\limits_a^{\omega} f(x, y)\,dx$ --- сходится равномерно на $[c, d]$

Тогда $F$ --- непрерывна на $[c, d]$ (следовательно, интегрируема) и 
\[ \int\limits_c^d dy \int\limits_a^{\omega} f(x, y)\,dx = \int\limits_a^{\omega} dx \int\limits_c^d f(x, y)\,dy \]
\begin{proof}
    Пусть $a < t < \omega$. Для собственного интеграла 
    \[ g(t, y) = \int\limits_a^t f(x, y)\,dx \]
    возможность внесения $\int\limits_c^d dy$ следует из непрерывности $f$:
    \[ \int\limits_c^d g(t, y)\,dy = \int\limits_c^d dy \int\limits_a^t f(x, y)\,dx = 
    \int\limits_a^t dx \underset{\text{непр. ф-ция}}{\underbrace{\int\limits_c^d f(x, y)\,dy}} \]
    \phantom{$\int\limits_c^d g(t, y)\,dy =\ \ \ \ \ $} $\overset{\downarrow}{\int\limits_c^d F(y)\,dy} \ \ \ \ \ = \ \ \ \ \ 
    \overset{\downarrow}{\int\limits_a^{\omega} dx \int\limits_c^d f(x, y)\,dy}$
    \[ \lim_{t \to \omega}\int\limits_c^d  dy \left( \int\limits_a^t f(x, y)\,dx \right) = 
    \int\limits_c^d \left( \lim_{t \to \omega} \int\limits_a^t f(x, y)\,dx \right) dy = \int\limits_c^d F(y)\,dy \]
    Имеем право на такой предельный переход, потому что $g(t, y)$ непрерывна.
\end{proof}

\subsubsection{Теорема о несобственном интегрировании по параметру под знаком несобственного интеграла.}
\[ \int\limits_c^{\tilde \omega} F(y)\,dy = \int\limits_c^{\tilde \omega} dy \int\limits_a^{\omega} f(x, y)\,dx \]
Пусть \\
\phantom{Пусть} $f(x, y)$ непрерывна на $[a, \omega) \times [c, \tilde \omega)$ \\
\phantom{Пусть} $F(y) = \int\limits_a^{\omega} f(x, y)\,dx$ --- сходится равномерно на $[c, \tau]$, 
где $c < \tau < \tilde \omega$, \\
\phantom{Пусть} $\Phi(x) = \int\limits_c^{\tilde \omega} f(x, y)\,dy$ --- сходится равномерно на $[a, t]$, 
где $a < t < \omega$, \\
\phantom{Пусть} хотя бы 1 из интегралов:
\[ \int\limits_c^{\tilde \omega} dy \int\limits_a^{\omega} |f(x, y)|\,dx, \ \ 
\int\limits_a^{\omega} dx \int\limits_c^{\tilde \omega} |f(x, y)|\,dy \]
\phantom{Пусть} сходится.

Тогда $\int\limits_c^{\tilde \omega} dy \int\limits_a^{\omega} f(x, y)\,dx = 
\int\limits_a^{\omega} dx \int\limits_c^{\tilde \omega} f(x, y)\,dy$

Заметим, что условия сформулированы симметрично относительно $x$ и $y$, как и 
утверждение самой теоремы. Поэтому когда мы говорим ``существует хотя бы один из интегралов'',
то мы можем договориться какой именно, потому что если что мы можем просто переобозначить
$x$ и $y$. Поэтому будем считать что сходится именно правый интеграл, где внутри 
$dy$ а снаружи $dx$.

\begin{proof}
    \[ \forall \tau \in (c; \tilde \omega) \ \ \ \int\limits_c^{\tau} dy \int\limits_a^{\omega} f(x, y)\,dx = 
    \int\limits_a^{\omega} dx \int\limits_c^{\tau} f(x, y)\,dy \text{ по \hyperref[subsubsec:6.2]{6.2}}\] 
    Рассмотрим предельный переход $\tau \to \tilde \omega$:
    \[ \lim_{\tau \to \tilde \omega} \int\limits_a^{\omega} dx \int\limits_c^{\tau} f(x, y)\,dy ~ \text{(1)}\]
    \[ \varphi(\tau, x) = \int\limits_c^{\tau} f(x, y)\,dy \overset{x \in [a; t]}{\underset{\tau \to \tilde \omega}
    {\rightrightarrows}} \Phi(x) = \int\limits_c^{\tilde \omega} f(x, y)\,dy \]
    --- по условию
    \[ |\varphi(\tau, x)| \le \int\limits_c^{\tilde \omega} |f(x, y)|\,dy, \ \ \int\limits_a^{\omega} |\varphi(\tau, x)|\,dx
    \le \int\limits_a^{\omega} dx \int\limits_c^{\tilde \omega} |f(x, y)|\,dy \]
    --- сходятся по условию \\
    $\Rightarrow$ по признаку Вейерштрасса $\int\limits_a^{\omega} \varphi(\tau, x)\,dx$ сходится равномерно

    С одной стороны $\int_a^\omega \varphi(\tau, x)\,dx$ сходится равномерно, а с другой стороны
    $\varphi(\tau, x) \overset{x \in [a; t]}{\underset{\tau \to \tilde \omega}
    {\rightrightarrows}} \int\limits_c^{\tilde \omega} f(x, y)\,dy$ по условию.
    Из этих двух пунктов вытекает, что знак предела по переменной $\tau$ мы можем внести внести внутрь
    интеграла по переменной $x$ в выражении (1). Так мы получим
    несобственный интеграл по $x$ от несобственного интеграла по $y$. Так мы с вами показали, что
    правая $\int\limits_a^{\omega} dx \int\limits_c^{\tau} f(x, y)\,dy \to \int_a^\omega\,dx\int_c^{\tilde \omega}
    f\,dy$, а $\int\limits_c^{\tau} dy \int\limits_a^{\omega} f(x, y)\,dx$ будет стремится к
    соответствующему несобственному интегралу просто по определению несобственного интеграла, а поскольку
    равенство между этими частями всегда сохраняется, значит равенство будет верно и в пределе.
\end{proof}
