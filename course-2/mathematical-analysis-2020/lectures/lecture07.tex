\section{Лекция 7 - 27.10.2020 - Мера Жордана}
\subsection{Мера на кольце множеств}

\begin{definition}
    Пусть $\mathcal{F}$ -- некоторое семейство подмножеств множества $X$, т.е. $\mathcal{F} \subseteq 2^X$. Функция $\mu$: $\mathcal{F} \to [0; +\infty)$ называется
    мерой на $\mathcal{F}$, если она обладает свойством аддитивности:
    $$\mu(A \sqcup B) = \mu(A) + \mu(B)$$
\end{definition}

Множество $\mathcal{F}$ называется кольцом, если:
\begin{enumerate}
    \item $\varnothing \in \mathcal{F}$
    \item Если $A$, $B$ $\in \mathcal{F}$, то $A \cup B$, $A \cap B$ и $A \setminus B$ содержатся в $\mathcal{F}$
\end{enumerate}

Свойства меры на кольце:
\begin{enumerate}
    \item $\mu(\varnothing) = 0$
    \item $A \subseteq B \implies \mu(A) \leq \mu(B)$
    \item $\mu(A \cup B) = \mu(A) + \mu(B) - \mu(A \cap B)$
\end{enumerate}

\subsection{Ограниченные полуинтервалы в $\RR^m$}

$\RR$: $[a; b)$

$\RR^2$: $[a; b) \times [c; d)$

$\RR^m$: $[a^1; b^1) \times [a^2; b^2) \times \dots \times [a^m; b^m)$, $a = (a^1, \dots, a^m), b = (b^1, \dots, b^m) \in \RR^m$

$[a; a) = \varnothing$


$\RR$:

Пересечение двух полуинтервалов -- полуинтервал.

Разность двух полуинтервалов -- полуинтервал или объединение двух непересекающихся полуинтервалов.

$\RR^n$:

Разность двух полуинтервалов есть объединение не более, чем $2m$ дизъюнктных полуинтервалов.

\subsection{Кольцо простых множеств}

\begin{definition}
    Простым множество называется объединением конечного числа полуинтервалов:
    $$E = \bigcup_{i=1}^{n} E_i = \bigcup_{i=1}^{n} [a_i; b_i)$$
\end{definition}

Простые множества образуют кольцо:

$\varnothing = [a; a)$ -- простое.

$E_1, E_2$ -- простые, то: $E_1 \cup E_2$ -- простое, $E_1 \cap E_2$ -- простое, $E_1 \setminus E_2$ -- простое.

$E = \bigcup_{i=1}^{n} E_i = E_i \sqcup (E_2 \setminus E_1) \sqcup (E_3 \setminus E_1 \setminus E_2 \sqcup \dots)$

$E$ представимо в виде объединения дизъюнктных полуинтервалов: $E = \bigsqcup_{j=1}^{m}[a_j; b_j)$

$\mu([a; b)) = (b^1 - a^1) \cdot (b^2 - a^2) \cdot \dots \cdot (b^m - a^m)$, где все $b^i \geq a^i$

$\mu(E) = \mu(\bigsqcup[a_j; b_j)) = \sum_{j=1}^{m} \mu([a_j; b_j))$

\subsection{Внешняя $m$-мерная мера Жордана}

$A \subset \RR^m$, $A$ -- ограниченное множество.

Внешней мерой Жордана множества $A$ называется $\overline{\mu}(A) = \inf_{E, A \subseteq E} \mu(E)$

Свойства внешней меры:

\begin{enumerate}
    \item $\overline{\mu}(\varnothing) = 0$
    \item $A \subseteq B \implies \overline{\mu}(A) \leq \overline{\mu}(B)$
    \begin{proof}
        Т.к. $\forall E, B \subseteq E \implies A \subseteq E$, т.е. при вычислении $\overline{\mu}(A)$ $\inf$ берётся по более широкому классу множеств $E$.
    \end{proof}
    \item $\overline{\mu}(A \cup B) \leq \overline{\mu}(A) + \overline{\mu}(B)$
    \begin{proof}
        $A \subseteq E_1, B \subseteq E_2 \implies A \cup B \subseteq E_1 \cup E_2$

        $\overline{\mu}(A \cup B) \leq \mu(E_1 \cup E_2) \leq \mu(E_1) + \mu(E_2) \leq \overline{\mu}(A) + \overline{\mu}(B)$
    \end{proof}
    \item Внешняя мера не обладает свойством аддитивности
\end{enumerate}

\subsection{Измеримость по Жордану}

\begin{definition}
    Ограниченное множество $A \subset \RR^m$ называется измеримым по Жордану, если $\forall \varepsilon > 0$ $\exists E, A \subseteq E$: $\overline{\mu}(E \setminus A) < \varepsilon$
\end{definition}

\begin{enumerate}
    \item $\varnothing$ -- измеримо
    \item $A$, $B$ -- измеримы $\implies A \cup B$, $A \cap B$, $A \setminus B$ -- измеримы
\end{enumerate}

Значит, измеримые множества образуют кольцо.

На кольце измеримых множеств внешняя мера аддитивна.

\begin{definition}
    Рассмотрим теперь $E \subseteq A$, тогда $\underline{\mu}(A) = \sup_{E \subseteq A} \mu(E)$ -- внутренняя мера $A$.
\end{definition}

Множество $A$ измеримо $\iff$ $\overline{\mu}(A) = \underline{\mu}(A)$

$\partial A$ -- граница множества $A$.

Если $E_1 \subseteq A \subseteq E_2$, то $\partial A \subseteq E_2 \setminus E_1$.

Если $A$ -- измеримо, то $\overline{\mu}(\partial A) = 0$

\subsection{Интегрируемость функции по Риману и измеримость по Жордану её подграфика}

$f(x) \geq 0$ на $[0; 1]$

$A = \{(x, y) | 0 \leq y \leq f(x), x \in [0; 1]\}$ -- подграфик функции $f$.

Функция $f$ интегрируема на $[0; 1]$ $\iff$ $A$ измеримо по Жордану, причём $\mu(A) = \int_{0}^{1} f(x) dx$