\section{Кратные интегралы. Брусы. Интегрируемые функции по Риману}

\subsection{Брус. Мера бруса}

\definition Замкнутый брус (координатный промежуток) в $\mathbb{R}^n$ — множество, описываемое как

\begin{equation*}
\begin{aligned}
    I&=\{x\in\mathbb{R}^n\ |\ a_i\leq x_i\leq q_i,\ i\in\{1,n\}\}\\
    &=\left[a_1,b_1\right]\times\ldots\times\left[a_n,b_n\right]
\end{aligned}
\end{equation*}

\comment $I=\{a_1,b_1\}\times\ldots\times\{a_n,b_n\}$, где $\{a_i, b_i\}$ может быть отрезком, интервалом и т.д.

\definition Мера бруса — его объём:

\begin{equation*}
    \begin{aligned}
        \mu(I)&=|I|
        =\prod_{i=1}^{n} (b_i-a_i)
    \end{aligned}
\end{equation*}

\subsection{Свойства меры бруса в $\R^n$}

\begin{enumerate}
    \item \textbf{Однородность:} $\mu(I_{\lambda a,\lambda b})=\lambda^n\cdot\mu(I_{a,b})$, где $\lambda\geq
    0$
    \item \textbf{Аддитивность:} Пусть $I, I_1, \ldots, I_k$ — брусы
    
    Тогда, если $\forall i, j\, I_i, I_j$ не имеют общих внтренних точек, и $\displaystyle\bigcup_{i=1}^kI_i = I$, то
    $$|I| = \sum_{i=1}^k|I_i|$$
    \item \textbf{Монотонность}: Пусть $I$ — брус, покрытый конечной системой брусов, то есть $I\subset \displaystyle\bigcup_{i=1}^kI_i$, тогда
    $$|I| < \sum_{i=1}^k|I_i|$$
\end{enumerate}

\subsection{Разбиение бруса. Диаметр множества. Масштаб разбиения}

\definition \label{1.3} $I$ — замкнутый, невырожденный брус и $\displaystyle\bigcup_{i=1}^kI_i = I$, где $I_i$ попарно не имеют общих внутренних точек. Тогда набор $\T = \{\T\}_{i=1}^k$ называется разбиением бруса $I$

\definition \label{1.4} Диаметр произвольного ограниченного множества $M\subset\R^n$ будем называть 

\begin{equation*}
\begin{aligned}
    d(M) = \displaystyle\sup_{1\leq i\leq k}\|x-y\|,\text{ где}\\
    \|x-y\|=\sqrt{\sum_{i=1}^{n}\left(x_i-y_i\right)^2}
\end{aligned}
\end{equation*}

\definition \label{1.5} Масштаб разбиения $\T=\{I_i\}_{i=1}^k$ — число $\lambda(\T) = \Delta_{\T} = \displaystyle\max_{1\le i\le k} d(I_i)$

\definition \label{1.6} Пусть $\forall\ I_i$ выбрана точка $\xi_i\in I_i$. Тогда, набор $\xi = \{\xi_i\}_{i=1}^k$ будем называть \textbf{отмеченными точками}

\definition \label{1.7} Размеченное разбиение — пара $(\T, \xi)$

\subsection{Интегральная сумма Римана. Интегрируемость по Риману}
Пусть $I$ — невырожденный, замкнутый брус, функция $f: I\rightarrow \R$ определена на $I$

\definition \label{1.8} Интегральная сумма Римана функции $f$ на $(\T, \xi)$ — величина
$$\sigma(f, \T, \xi) := \sum_{i=1}^kf(\xi_i)\cdot|I_i|$$

\definition \label{1.9} Функция $f$ интегрируема (по Риману) на замкнутом брусе $I$ ($f:I\rightarrow\R$), если 

\begin{equation*}
\begin{aligned}
    \exists A\in\R: \forall \varepsilon > 0\, \exists \delta > 0: \forall(\T, \xi): \Delta_{\T} < \delta:\\
    |\sigma(f, \T, \xi)| - A| < \varepsilon
\end{aligned}
\end{equation*}

Тогда 
$$A = \int\limits_If(x)\d{x} = \underset{I}{\int\ldots\int}f(x_1, \ldots, x_n)\d{x_1}\ldots \d{x_n}$$
Обозначение: $f\in\mathcal{R}(I)$

\subsection{Пример константной функции}

Пусть у нас есть функция $f = \text{const}$
\begin{equation*}
\begin{aligned}
    \forall(\T, \xi):\ \sigma(f, \T, \xi)&= \sum_{i = 1}^k \text{const}\cdot|I_i|\\
    &= \text{const}\cdot|I| \Longrightarrow \int_I f(x)\d{x} = \text{const}\cdot|I|
    \end{aligned}
\end{equation*}

\subsection{Неинтегрируемая функция}

Имеется брус $I = [0, 1]^n$, а также определена функция, такая что
\begin{equation*}
    f = \begin{cases}
        1,& \forall i = \overline{1,\ldots, n}\,\, x_i\in \mathbb{Q}\\
        0,&\text{иначе}
    \end{cases}
\end{equation*}

\proof $\forall \T$ можно выбрать $\xi_i\in \mathbb{Q}$, тогда для такой пары $(\T, \overline{\xi})$:

\begin{equation*}
    \sigma(f, \T, \overline{\xi}) = \sum_{i=1}^k1\cdot|I_i| = |I| = 1
\end{equation*}

В то же время, $\forall \T$ можно выбрать $\xi_i\notin \mathbb{Q}$, тогда для такой пары $(\T, \hat{\xi})$:
\begin{equation*}
    \sigma(f, \T, \hat{\xi}) = \sum_{i=1}^k0\cdot|I_i| = 0 \Longrightarrow f\notin\mathcal{R}(I)
\end{equation*}

\subsection{Вычисление многомерного интеграла}

Вычислите интеграл
$$\iint\limits_{\substack{0\leq x\leq 1\\ 0\leq y\leq 1}}xy\d{x}\d{y}$$
рассматривая его как представление интегральной суммы при сеточном разбиении квадрата $$I = [0, 1]\times[0, 1]$$ на ячейки — квадраты со сторонами, длины которых равны $\frac{1}{n}$, выбирая в качестве точек $\xi_i$ нижние правые вершины ячеек

\begin{minipage}{0.5\textwidth}
Имеется функция $f = xy,\ |I| =\displaystyle\frac{1}{n^2}$
\begin{equation*}
    \begin{aligned}
        \sigma(f, \T, \xi) &= \sum_{i=1}^n \sum_{j=0}^{n-1}\frac{i}{n}\cdot\frac{j}{n}\cdot\frac{1}{n^2}\\
        &= \frac{1}{n^4}\sum_{i=1}^n\sum_{j=0}^{n-1} i\cdot j\\
        &= \frac{1}{n^4}\sum_{i=1}^n i \sum_{j=0}^{n-1} j\\
        &= \frac{n (n-1) }{n^4} \sum_{i=1}^ni\\
        &= \frac{n^2 (n+1) (n-1)}{4n^4}
        % \underset{n\to\infty}{\longrightarrow}\frac{1}{4}
    \end{aligned}
\end{equation*}
Заметим, что $\lim\limits_{n\rightarrow\infty}\displaystyle \frac{n^2 (n+1) (n-1)}{4n^4}=\frac{1}{4}$
\end{minipage}
\begin{minipage}{0.5\textwidth}
$$
    \begin{tikzpicture}[scale=3]
        \draw[step=0.25cm, gray, very thin] (0,0) grid (2,2);

        \draw[thick] (0,0) rectangle (2,2);
        
        \node at (-0.1, -0.1) {$0$};
        \node at (1, -0.3) {\text{$n$ штук}};
        \node at (2.1, -0.1) {$1$};
        \node at (-0.1, 2) {$1$};

        \foreach \i in {0.25, 0.5, ..., 2} {
        \foreach \j in {0, 0.25, ..., 1.75} {
            \fill (\i,\j) circle (1pt);
        }
    }

    \end{tikzpicture}
$$
\end{minipage}